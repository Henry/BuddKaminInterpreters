\documentclass[10pt,twoside,a4paper]{report}

\usepackage{a4}

\setlength{\textwidth}{5.5in}
\setlength{\oddsidemargin}{0.77in}
\setlength{\evensidemargin}{0in}
\setlength{\topmargin}{0in}

\usepackage{url}

\usepackage[
    plainpages=false,
    pdfpagelabels,
    bookmarksnumbered,
    hyperindex=true
]{hyperref}

\usepackage[usenames,dvipsnames]{color}

\hypersetup{
    bookmarks=true,         % show bookmarks bar?
    unicode=false,          % non-Latin characters in Acrobat’s bookmarks
    pdftoolbar=true,        % show Acrobat’s toolbar?
    pdfmenubar=true,        % show Acrobat’s menu?
    pdffitwindow=false,     % window fit to page when opened
    pdfstartview={FitH},    % fits the width of the page to the window
    pdftitle={My title},    % title
    pdfauthor={Author},     % author
    pdfsubject={Subject},   % subject of the document
    pdfcreator={Creator},   % creator of the document
    pdfproducer={Producer}, % producer of the document
    pdfkeywords={keywords}, % list of keywords
    pdfnewwindow=true,      % links in new window
    colorlinks=true,        % false: boxed links; true: colored links
    linkcolor=red,          % color of internal links
    citecolor=green,        % color of links to bibliography
    filecolor=magenta,      % color of file links
    urlcolor=cyan           % color of external links
}

\usepackage{listings}

% \lstset{% general command to set parameter(s)
%     language=C++,
%     basicstyle=\small\ttfamily, % print whole listing small
%     % basicstyle=\footnotesize\ttfamily,
%     keywordstyle=\bfseries\color{OliveGreen},
%     commentstyle=\itshape\color{Purple},
%     identifierstyle=\color{blue},
%     stringstyle=\color{Orange},
%     showstringspaces=false,      % no special string spaces
%     captionpos=b
% }
\lstset{% general command to set parameter(s)
    language=C++,
    basicstyle=\ttfamily\small,
    identifierstyle=\sffamily\color{blue},
    keywordstyle=\sffamily\bfseries\color{OliveGreen},,
    commentstyle=\rmfamily\color{Purple},
    stringstyle=\rmfamily\itshape\color{Orange},
    numberstyle=\scriptsize,
    showstringspaces=false,
    showstringspaces=false,      % no special string spaces
    captionpos=b
}

\newcommand{\includecode}[3] { %
\lstinputlisting[
label=#2,
firstline=\csname #2firstline\endcsname,
lastline=\csname #2lastline\endcsname,
float,
caption=#3]
{../Src/#1}
}

\begin{document}

\title{The Kamin Interpreters in C++}
\author{Tim Budd}
\maketitle
\cleardoublepage

\begin{abstract}
    This paper describes a series of interpreters for the languages used in the
    book ``Programming Languages: An Interpreter-Based Approach'' by Samuel
    Kamin (Addison-Wesley, 1989).  Unlike the interpreters provided by Kamin,
    which are written in Pascal, these interpreters are written in C++.  It is
    my belief that the use of inheritance in C++ better illustrates the unique
    features of each of the several languages.  In the Pascal versions of the
    interpreters the differences between the various interpreters, although
    small, are scattered throughout the code.  In the C++ versions differences
    are produced using only the mechanism of subclassing.  This means that the
    vast majority of code remains the same, and differences can be much more
    precisely isolated.

    The chapters in this report correspond to the chapters in the original text.
    Where motivational or background material is provided in that source it is
    generally omitted here.  A major exception is in those places (chiefly
    chapters 3, 7 and 8) where I have selected a syntax slightly different from
    that provided by Kamin.

    The use of an Object-Oriented language for the interpreters may seem a bit
    incongruous, since Object-Oriented programming is not discussed until
    Chapter 7.  Nevertheless, I think the benefits of programming the
    interpreters in C++ outweighs this problem.
\end{abstract}

\tableofcontents
\listoffigures
\lstlistoflistings

\newcommand*{\Environmentfirstline}{14}
\newcommand*{\Environmentlastline}{43}
\newcommand*{\Exprfirstline}{9}
\newcommand*{\Expressionfirstline}{49}
\newcommand*{\IntegerExpressionfirstline}{83}
\newcommand*{\Symbolfirstline}{107}
\newcommand*{\Exprlastline}{34}
\newcommand*{\Expressionlastline}{79}
\newcommand*{\IntegerExpressionlastline}{103}
\newcommand*{\Symbollastline}{128}
\newcommand*{\Functionfirstline}{9}
\newcommand*{\IntegerBinaryFunctionfirstline}{66}
\newcommand*{\BooleanBinaryFunctionfirstline}{83}
\newcommand*{\UserFunctionfirstline}{100}
\newcommand*{\Functionlastline}{22}
\newcommand*{\IntegerBinaryFunctionlastline}{79}
\newcommand*{\BooleanBinaryFunctionlastline}{96}
\newcommand*{\UserFunctionlastline}{114}
\newcommand*{\LispReaderfirstline}{12}
\newcommand*{\BooleanUnaryfirstline}{73}
\newcommand*{\Definefirstline}{103}
\newcommand*{\IfStatementfirstline}{113}
\newcommand*{\WhileStatementfirstline}{123}
\newcommand*{\SetStatementfirstline}{133}
\newcommand*{\BeginStatementfirstline}{143}
\newcommand*{\LispReaderlastline}{36}
\newcommand*{\BooleanUnarylastline}{86}
\newcommand*{\Definelastline}{109}
\newcommand*{\IfStatementlastline}{119}
\newcommand*{\WhileStatementlastline}{129}
\newcommand*{\SetStatementlastline}{139}
\newcommand*{\BeginStatementlastline}{149}
\newcommand*{\Listfirstline}{9}
\newcommand*{\Listlastline}{61}
\newcommand*{\ReaderClassfirstline}{11}
\newcommand*{\ReaderClasslastline}{39}
\newcommand*{\BasicLispIsTruefirstline}{15}
\newcommand*{\BasicLispInitializefirstline}{28}
\newcommand*{\BasicLispIsTruelastline}{24}
\newcommand*{\BasicLispInitializelastline}{51}
\newcommand*{\LispIsTruefirstline}{16}
\newcommand*{\LispInitializefirstline}{27}
\newcommand*{\LispIsTruelastline}{23}
\newcommand*{\LispInitializelastline}{67}
\newcommand*{\APLValuefirstline}{32}
\newcommand*{\APLreaderfirstline}{154}
\newcommand*{\readAPLscalarfirstline}{198}
\newcommand*{\APLScalarFunctionApplyfirstline}{343}
\newcommand*{\APLReductionfirstline}{419}
\newcommand*{\APLCompressionFunctionApplyfirstline}{479}
\newcommand*{\APLShapeFunctionApplyfirstline}{548}
\newcommand*{\APLRavelFunctionApplyfirstline}{583}
\newcommand*{\APLRestructFunctionApplyfirstline}{608}
\newcommand*{\APLIndexFunctionApplyfirstline}{648}
\newcommand*{\APLCatenationFunctionApplyfirstline}{678}
\newcommand*{\APLTransposeFunctionApplyfirstline}{761}
\newcommand*{\APLSubscriptFunctionfirstline}{805}
\newcommand*{\APLInitializefirstline}{833}
\newcommand*{\APLValuelastline}{65}
\newcommand*{\APLreaderlastline}{194}
\newcommand*{\readAPLscalarlastline}{235}
\newcommand*{\APLScalarFunctionApplylastline}{396}
\newcommand*{\APLReductionlastline}{463}
\newcommand*{\APLCompressionFunctionApplylastline}{532}
\newcommand*{\APLShapeFunctionApplylastline}{567}
\newcommand*{\APLRavelFunctionApplylastline}{592}
\newcommand*{\APLRestructFunctionApplylastline}{632}
\newcommand*{\APLIndexFunctionApplylastline}{662}
\newcommand*{\APLCatenationFunctionApplylastline}{745}
\newcommand*{\APLTransposeFunctionApplylastline}{789}
\newcommand*{\APLSubscriptFunctionlastline}{829}
\newcommand*{\APLInitializelastline}{872}
\newcommand*{\SchemeLambdaFunctionfirstline}{31}
\newcommand*{\SchemeInitializefirstline}{59}
\newcommand*{\SchemeLambdaFunctionlastline}{55}
\newcommand*{\SchemeInitializelastline}{99}
\newcommand*{\SASLThunkfirstline}{29}
\newcommand*{\SASLThunkPredicatesfirstline}{79}
\newcommand*{\SASLThunkTouchfirstline}{120}
\newcommand*{\SASLThunkEvalfirstline}{142}
\newcommand*{\SaslConsFunctionfirstline}{155}
\newcommand*{\SASLLazyFunctionfirstline}{183}
\newcommand*{\SASLThunklastline}{75}
\newcommand*{\SASLThunkPredicateslastline}{116}
\newcommand*{\SASLThunkTouchlastline}{138}
\newcommand*{\SASLThunkEvallastline}{146}
\newcommand*{\SaslConsFunctionlastline}{175}
\newcommand*{\SASLLazyFunctionlastline}{235}
\newcommand*{\CLUClusterfirstline}{31}
\newcommand*{\CLUSelectorModifierfirstline}{94}
\newcommand*{\CLUClusterDeffirstline}{168}
\newcommand*{\CLUClusterlastline}{90}
\newcommand*{\CLUSelectorModifierlastline}{164}
\newcommand*{\CLUClusterDeflastline}{297}
\newcommand*{\SmalltalkObjectfirstline}{26}
\newcommand*{\SmalltalkObjectApplyfirstline}{88}
\newcommand*{\SmalltalkObjectGetNamesfirstline}{142}
\newcommand*{\SmalltalkIntegerfirstline}{176}
\newcommand*{\SmalltalkSymbolfirstline}{241}
\newcommand*{\SmalltalkIfMethodfirstline}{274}
\newcommand*{\SmalltalkReaderfirstline}{318}
\newcommand*{\SmalltalkNewMethodDoMethodfirstline}{365}
\newcommand*{\SmalltalkSubclassMethodfirstline}{405}
\newcommand*{\SmalltalkMethodMethodDoMethodfirstline}{457}
\newcommand*{\SmalltalkInitializefirstline}{499}
\newcommand*{\SmalltalkObjectlastline}{84}
\newcommand*{\SmalltalkObjectApplylastline}{138}
\newcommand*{\SmalltalkObjectGetNameslastline}{152}
\newcommand*{\SmalltalkIntegerlastline}{202}
\newcommand*{\SmalltalkSymbollastline}{255}
\newcommand*{\SmalltalkIfMethodlastline}{302}
\newcommand*{\SmalltalkReaderlastline}{348}
\newcommand*{\SmalltalkNewMethodDoMethodlastline}{389}
\newcommand*{\SmalltalkSubclassMethodlastline}{434}
\newcommand*{\SmalltalkMethodMethodDoMethodlastline}{495}
\newcommand*{\SmalltalkInitializelastline}{537}
\newcommand*{\PrologValuefirstline}{22}
\newcommand*{\PrologEvalfirstline}{88}
\newcommand*{\PrologContiuationfirstline}{161}
\newcommand*{\PrologComposeContinuationfirstline}{188}
\newcommand*{\PrologAndContinuationfirstline}{224}
\newcommand*{\PrologOrContinuationfirstline}{260}
\newcommand*{\PrologUnifyfirstline}{302}
\newcommand*{\PrologUnifyContinuationfirstline}{349}
\newcommand*{\PrologPrintContinuationfirstline}{404}
\newcommand*{\PrologUnifyOperationfirstline}{444}
\newcommand*{\PrologQueryStatementfirstline}{495}
\newcommand*{\PrologInitializefirstline}{542}
\newcommand*{\PrologValuelastline}{62}
\newcommand*{\PrologEvallastline}{129}
\newcommand*{\PrologContiuationlastline}{183}
\newcommand*{\PrologComposeContinuationlastline}{220}
\newcommand*{\PrologAndContinuationlastline}{256}
\newcommand*{\PrologOrContinuationlastline}{298}
\newcommand*{\PrologUnifylastline}{345}
\newcommand*{\PrologUnifyContinuationlastline}{396}
\newcommand*{\PrologPrintContinuationlastline}{436}
\newcommand*{\PrologUnifyOperationlastline}{487}
\newcommand*{\PrologQueryStatementlastline}{538}
\newcommand*{\PrologInitializelastline}{561}
\newcommand*{\EnvironmentAddfirstline}{32}
\newcommand*{\EnvironmentLookupfirstline}{68}
\newcommand*{\EnvironmentAddlastline}{64}
\newcommand*{\EnvironmentLookuplastline}{88}
\newcommand*{\ExprAssignfirstline}{12}
\newcommand*{\ExprAssignlastline}{42}
\newcommand*{\FunctionApplyfirstline}{31}
\newcommand*{\IntegerBinaryFunctionApplyfirstline}{108}
\newcommand*{\BooleanBinaryFunctionApplyfirstline}{129}
\newcommand*{\UserFunctionApplyfirstline}{172}
\newcommand*{\FunctionApplylastline}{47}
\newcommand*{\IntegerBinaryFunctionApplylastline}{121}
\newcommand*{\BooleanBinaryFunctionApplylastline}{144}
\newcommand*{\UserFunctionApplylastline}{195}
\newcommand*{\LispReaderImplfirstline}{23}
\newcommand*{\IntegerArithmeticFunctionsfirstline}{53}
\newcommand*{\EqualFunctionfirstline}{84}
\newcommand*{\IntegerRelationalFunctionsfirstline}{120}
\newcommand*{\CarCdrConsfirstline}{141}
\newcommand*{\BooleanUnaryApplyfirstline}{174}
\newcommand*{\DefineApplyfirstline}{260}
\newcommand*{\IfStatementApplyfirstline}{292}
\newcommand*{\WhileStatementApplyfirstline}{315}
\newcommand*{\SetStatementApplyfirstline}{346}
\newcommand*{\BeginStatementApplyfirstline}{371}
\newcommand*{\LispReaderImpllastline}{45}
\newcommand*{\IntegerArithmeticFunctionslastline}{76}
\newcommand*{\EqualFunctionlastline}{116}
\newcommand*{\IntegerRelationalFunctionslastline}{133}
\newcommand*{\CarCdrConslastline}{166}
\newcommand*{\BooleanUnaryApplylastline}{216}
\newcommand*{\DefineApplylastline}{285}
\newcommand*{\IfStatementApplylastline}{311}
\newcommand*{\WhileStatementApplylastline}{342}
\newcommand*{\SetStatementApplylastline}{367}
\newcommand*{\BeginStatementApplylastline}{385}
\newcommand*{\ListEvalfirstline}{77}
\newcommand*{\ListEvallastline}{120}
\newcommand*{\mainfirstline}{25}
\newcommand*{\mainlastline}{55}
\newcommand*{\ReaderPromptAndReadfirstline}{93}
\newcommand*{\ReaderReadExpressionfirstline}{116}
\newcommand*{\ReaderPromptAndReadlastline}{112}
\newcommand*{\ReaderReadExpressionlastline}{156}

\chapter{The Basic Interpreter}

The structure of our basic interpreter\footnote{It should be noted that our
    basic interpreter is not an interpreter for Basic.}  differs somewhat from
that described by Kamin.  Our interpreter is structured around a small main
program which manipulates three distinct types of data structures.  The main
program is shown in Figure~\ref{main}, and will be discussed in more detail in
the next section.  Each of the three main data structures is represented by a
C++ class, such is subclassed in various ways by the different interpreters.
The three varieties of data structures are the following:
%
\begin{itemize}
    \item {\bf Readers}.  Instances of this class prompt the user for input
    values, and break the input into a structure of unevaluated components.  A
    single instance of either the class {\sf Reader} or a subclass is created
    during the initialization process for each interpreter.  The base reader
    class is subclassed in those interpreters which introduce new syntactic
    elements (such as quoted lists in Lisp or vectors in APL).
    \item {\bf Environments}.  An Environment is a data structure used to
    maintain a collection of symbol-value pairs, such as the global run-time
    environment or the values of arguments passed to a function.  Values can be
    added to an environment, and the existing binding of a symbol to a value can
    be changed to a new value.
    \item {\bf Expressions}.  Expressions represent the heart of the system, and
    the differences between the various interpreters is largely found in the
    various different types of expressions they manipulate.  Expressions know
    how to ``evaluate themselves'' where the meaning of that expression is
    determined by each type of expression.  In addition expressions also know
    how to print their value, and that, too, differs for each type of
    expression.
\end{itemize}

In subsequent sections we will explore in more detail each of these data
structures.

\section{The Main Program}

Figure~\ref{main} shows the main program, \footnote{I have omitted the
    ``include'' directives and certain global declarations from this figure.
    The complete code can be found in ../Src/main.C.}  which defines the top
level control for the interpreters.  The same main program is used for each of
the interpreters.  Indeed, the vast majority of code remains constant throughout
the interpreters.
%
\includecode{main.C}{main}
{The Read-Eval-Print Loop for the interpreters}

The structure of the main program is very simple.  To begin, a certain amount of
initialization is necessary.  There are four global variables found in all the
interpreters.  The variable {\sf emptyList} contains a list with no elements.
(We will return to a discussion of lists in Section~\ref{listsec}).  The three
environments {\sf globalEnvironment}, {\sf valueOps} and {\sf commands}
represent the top-level context for the interpreters.  The {\sf
    globalEnvironment} contains those symbols that are accessible at the top
level.  The {\sf valueOps} are those operations that can be performed at any
level, but which are not symbols themselves that can be manipulated by the user.
Finally {\sf commands} are those functions that can be invoked only at the top
level of execution.  That is, commands cannot be executed within function
definitions.

Following the common initialization the function {\sf initialize} is called to
provide interpreter-specific initialization.  This chiefly consists of adding
values to the three environments.  This function is changed in each of the
various interpreters.

The heart of the system is a single loop, which executes until the user types
the directive {\sf quit}.\footnote{The reader data structure will trap
    end-of-input signals, and if detected acts as if the user had typed the {\sf
        quit} directive.}  The reader (which must be defined as part of the
interpreter-specific initialization) requests a value from the user.  After
testing for the the {\sf quit} directive, the entered expression is evaluated.
We will defer an explaination of the {\sf evalAndPrint} method until
Section~\ref{exprsec}, merely noting here that it evaluates the expression the
user has entered and prints the result.  The read-eval-print cycle then
continues.

\section{Readers}

Readers are implemented by instances of class {\sf Reader}, shown in
Figure~\ref{ReaderClass}.  The only public function performed by this class is
provided by the method {\sf promptAndRead}, which prints the interpreter prompt,
waits for input from the user, and then parses the input into a legal, but
unevaluated, expression (usually a symbol, integer or list-expression).  These
actions are implemented by a variety of utility routines, which are declared as
{\sf protected} so that they may be made available to later subclasses.
%
\includecode{reader.h}{ReaderClass}
{Class Description of the reader class}
\includecode{reader.C}{ReaderPromptAndRead}
{The Method {\sf promptandRead} from class {\sf Reader}}

The code that implements this data structure is relatively straight-forward, and
most of it will not be presented here.  The main method is the single
public-accessible routine {\sf promptAndRead}, which is shown in
Figure~\ref{ReaderPromptAndRead}.  This method loops until the user enters an
expression.  The method {\sf fillInputBuffer} places the instance pointer {\sf
    p} at the first non-space character (also stripping out comments).  Thus
lines containing only spaces, newlines, or comments are handled quickly here,
and cause no further action.  Also, as have noted previously, an end-of-input
indication is caught by the method {\sf fillInputBuffer}, which then places the
{\sf quit} command in the input buffer.  The method {\sf readExpression}
(Figure~\ref{ReaderReadExpression}) is the parser used to break the input into
an unevaluated expression.  This method is declared {\sf virtual}, and thus can
be redefined in subclasses.  The base method recognizes only integers, symbols,
and lists.  The routine to read a list recursively calls the method to read an
expression.
%
\includecode{reader.C}{ReaderReadExpression}
{The method {\sf readExpression} and {\sf readList}}

\section{Environments}

As we have noted already, the {\sf Environment} data structure is used to
maintain symbol-value pairings.  In addition to the global environments defined
during initialization, environments are created for argument lists passed to
functions, and in various other contexts by some of the later interpreters.
Environments can be linked together, so that if a symbol is not found in one
environment another can be automatically searched.  This facilitates lexical
scoping, for example.

For reasons having to do with memory management, the {\sf Environment} data
structure, shown in Figure~\ref{Environment}, is declared as a subclass of the
class {\sf Expression}.  Unlike other expressions, however, environments are
never directly manipulated by the user.  Also for memory management reasons,
there is a class {\sf Env} declared which can maintain a pointer to an
environment.  The two methods defined in class {\sf Env} set and return this
value.  Anytime a pointer is to be maintained for any period of time, such as
the link field in an environment, it is held in a variable declared as {\sf Env}
rather than as a pointer directly.  Finally the overridden virtual methods {\sf
    isEnvironment} and {\sf free} in class {\sf Environment} are also related to
memory management, and we will defer a discussion of these until the next
section.
%
\includecode{environment.h}{Environment}
{The {\sf Environment} data structure}

The three methods used to manipulate environments are {\sf lookup}, {\sf add}
and {\sf set}.  The first attempts to find the value of the symbol given as
argument, returning a null pointer if no value exists.  The method {\sf add}
adds a new symbol-value pair to the front of the current environment.  The
method {\sf set} is used to redefine an existing value.  If the symbol is not
found in the current environment and there is a valid link to another
environment the linked environment is searched.  If the link field is null (that
is, there is no next environment), the symbol and valued are {\sf add}ed to the
current environment.

Environments are implemented using the List data structure, a form of Expression
we will describe in more detail in Section~\ref{listsec}.  Two parallel lists
contain the symbol keys and their associated values.  For the moment it is only
necessary to characterize lists by four operations.  A list is composed of list
nodes (elements of class {\sf ListNode}).  Each node contains an expression (the
head) and, recursively, another list.  The special value {\sf emptyList}, which
we have already encountered, terminates every list.  The operation {\sf head}
returns the first element of a list node.  When provided with an argument, the
operation {\sf head} can be used to modify this first element.  The operation
{\sf tail} returns the remainder of the list.  Finally the operation {\sf isNil}
returns true if and only if the list is the empty list.
%
\includecode{environment.C}{EnvironmentLookup}
{The method {\sf lookup} in class {\sf Environment}}

Figure~\ref{EnvironmentLookup} shows the method {\sf lookup}, which is defined
in terms of these four operations.  The while loop cycles over the list of keys
until the end (empty list) is reached.  Each key is tested against the argument
key, using the equality test provided by the class {\sf Symbol}.  Once a match
is found the associated value is returned.

If the entire list of names is searched with no match found, if there is a link
to another environment the lookup message is passed to that environment.  If
there is no link, a null value is returned.

The routine to add a new value to an environment (Figure~\ref{EnvironmentAdd})
merely attaches a new name and value to the beginning of the respective lists.
Note by attaching to be beginning of a list this will hide any existing binding
of the name, although such a situation will not often occur.  The method {\sf
    set} searches for an existing binding, replacing it if found, and only
adding the new element to the final environment if no binding can be located.
%
\includecode{environment.C}{EnvironmentAdd}
{Methods used to Insert into an environment}

\section{Expressions}
\label{exprsec}

The class {\sf Expression} is a root for a class hierarchy that contains the
majority of classes defined in these interpreters.  Figure~\ref{classpic} shows
a portion of this class hierarchy.  We have already seen that environments are a
form of expression, as are integers, symbols, lists and functions.

\setlength{\unitlength}{5mm}
\begin{figure}
\begin{picture}(16,10)(-4,-3)
\put(-3.5,0){\sf Expression}
\put(0,0.2){\line(1,0){1}}
\put(1,0){\sf Function}
\put(0,0.2){\line(1,1){1}}
\put(1,1){\sf List}
\put(0,0.2){\line(1,2){1}}
\put(1,2){\sf Symbol}
\put(0,0.2){\line(1,3){1}}
\put(1,3){\sf Integer}
\put(0,0.2){\line(1,-2){1}}
\put(1,-2){\sf Environment}
\put(4,0.2){\line(1,0){1}}
\put(5,0){\sf BinaryFunction}
\put(4,0.2){\line(1,1){1}}
\put(5,1){\sf UnaryFunction}
\put(4,0.2){\line(1,2){1}}
\put(5,2){\sf BeginStatement}
\put(4,0.2){\line(1,3){1}}
\put(5,3){\sf SetStatement}
\put(4,0.2){\line(1,4){1}}
\put(5,4){\sf WhileStatement}
\put(4,0.2){\line(1,5){1}}
\put(5,5){\sf IfStatement}
\put(4,0.2){\line(1,6){1}}
\put(5,6){\sf DefineStatement}
\put(4,0.2){\line(1,-1){1}}
\put(5,-1){\sf UserFunction}
\put(9.5,0.2){\line(1,0){1}}
\put(10.5,0){\sf IntegerBinaryFunction}
\end{picture}
\caption{The {\sf Expression} class Hierarchy in Chapter 1}
\label{classpic}
\end{figure}

\subsection{The Abstract Class}

The major purposes of the abstract class {\sf Expression}
(Figure~\ref{Expression}) are to perform memory management functions, to permit
conversions from one type to another in a safe manner, and to define protocol
for evaluation and printing of expression values.  The latter is easist to
dismiss.  The virtual methods {\sf eval} and {\sf print} provide for evaluation
and printing of values.  The {\sf eval} method takes as argument a target
expression to which the evaluated expression will be assigned, as well as two
environments.  The first environment contains the list of legal value-ops for
the expression, while the second is the more general environment in which the
expression is to be evaluated.  The default method for {\sf eval} merely assigns
the current expression to the target.  This suffices for objects, such as
integers, which yield themselves no matter how many times they are evaluated.
The default method {\sf print}, on the other hand, prints an error message.
Thus this method should always be overridden in subclasses.
%
\includecode{expression.h}{Expression}
{The Class {\sf Expression}}

\subsubsection{Memory Management}

For long running programs it is imperative that memory associated with unused
expressions be recovered by the underlying operating system.  This is
accomplished in these interpreters through the mechanism of reference counts.
Every expression contains a reference count field, which is initially set to
zero by the constructor in class {\sf Expression}.  The integer value maintained
in this field represents the number of pointers that reference the object.  When
this count becomes zero, no pointers refer to the object and the memory
associated with it can be recovered.

The maintenance of reference counts if peformed by the class {\sf Expr}
(Figure~\ref{Expr}).  As with the class {\sf Env} we have already encountered,
the class {\sf Expr} is a holder class, which maintains an expression pointer.
A value can be inserted into an {\sf Expr} either through construction or the
assignment operator.  A value can be retrieved either though the protected
method {\sf val} or, as a notational convenience, through the parenthesis
operator.  The method {\sf evalAndPrint}, as have noted already, merely passes
the {\sf eval} message on to the underlying expression and prints the resulting
value.
%
\includecode{expression.h}{Expr}
{The class {\sf Expr}}

Figure~\ref{ExprAssign} gives the implementation of the constructor and
assignment operator for class {\sf Expr}.  The constructor takes an optional
pointer to an expression, which may be a null expression (the default).  If the
expression is non-null, the reference count for the expression is incremented.
Similarly, the assignment operator first increments the reference count of the
new expression.  Then it decrements the reference count of the existing
expression (if non-null), and if the reference count reaches zero, the memory is
released, using the system function {\sf delete}.  Immediately prior to
destruction, the virtual method {\sf free} is invoked.  Classes can override
this method to provide any necessary class-specific maintenance.  For example,
the class {\sf Environment} (Figure~\ref{Environment}) assigns null values to
the structures {\sf theNames}, {\sf theValues} and {\sf theLink}, thereby
possibly triggering the release of their storage as well.
%
\includecode{expression.C}{ExprAssign}
{Assignment and Initialization of Expressions}

\subsubsection{Type Conversion}

A common difficulty in a statically typed language such as C++ is the {\em
    container problem}.  Elements placed into a general purpose data structure,
such as a list, must have a known type.  Generally this is accomplished by
declaring such elements as a general type, such as {\sf Expression}.  But in
reality such elements are usually instances of a more specific subclass, such as
an integer or a symbol.  When we remove these values from the list, we would
like to be able to recover the original type.

There are actually two steps in the solution of this problem.  The first step is
testing the type of an object, to see if it is of a certain form.  The second
step is to legally assign the object to a variable declared as the more specific
class.  In these interpreters the mechanism of virtual methods is used to
combine these two functions.  In the abstract class {\sf Expression} a number of
virtual functions are defined, such as {\sf isInteger} and {\sf isEnvironment}.
These are declared as returning a pointer type.  The default behavior, as
provided by class {\sf Expression}, is to return a null pointer.  In an
appropriate class, however, this method is overridden so as to return the
current element.  That is, the class associated with integers overrides {\sf
    isInteger}, the class associated with symbols overrides {\sf isSymbol}, and
so on.  Figure~\ref{isEnvironment} shows the two definitions of {\sf
    isEnvironment}, the first from class {\sf Expression} and the second from
class {\sf Environment}.  By testing whether the result of this method is
non-null or not, one can not only test the type of an object but one can assign
the value to a specific class pointer without compromising type safety.  An
example bit of code is provided in Figure~\ref{isEnvironment} that illustrates
the use of these functions.

\begin{figure}
\begin{lstlisting}
Environment* Expression::isEnvironment()
{
    return 0;
}

Environment* Environment::isEnvironment()
{
    return this;
}

Expression* a = new Symbol("test");
Expression* b = new Environment(emptyList, emptyList, 0);

Environment* c = a->isEnvironment();   // will yield null
Environment* d = b->isEnvironment();   // will yield the environment

if (c)
{
    cout<< "c is an environment";  // won't happen
}
if (d)
{
    cout<< "d is an environment";  // will happen
}
\end{lstlisting}
\caption{Type safe object test and conversion}
\label{isEnvironment}
\end{figure}

The method {\sf touch} presents a slightly different situation.  It is defined
in the abstract class to merely return the object to which the message is sent.
That is, it is a null-operation.  In Chapter~\ref{sasl}, when we introduce
delayed evaluation, we will define a type of expression which is not evaluated
until it is needed.  This expression will override the touch method to force
evaluation at that point.

\subsection{Integers}

Internally within the interpreters integers are represented by the class {\sf
    IntegerExpression} (Figure~\ref{IntegerExpression}).  The actual integer
value is maintained as a private value set as part of the construction process.
This value can be accessed via the method {\sf val}.  The only overridden
methods are the {\sf print} method, which prints the integer value, and the {\sf
    isInteger} method, which yields the current object.
%
\includecode{expression.h}{IntegerExpression}
{The class {\sf IntegerExpression}}

\subsection{Symbols}

Symbols are used to represent uninterpreted character strings, for example
identifier names.  Instances of class {\sf Symbol} (Figure~\ref{Symbol})
maintain the text of their value in a private instance variable.  This character
pointer can be recovered via the method {\sf chars}.  Storage for this text is
allocated as part of the construction process, and deleted by the virtual method
{\sf free}.  The equality testing operators return true if the current symbol
matches the text of the argument.
%
\includecode{expression.h}{Symbol}
{The class {\sf Symbol}}

Figure~\ref{Symbol} also shows the implementation of the method {\sf eval} in
the class {\sf Symbol}.  When a symbol is evaluated it is used as a key to index
the current environment.  If found the (possibly touched) associated value is
assigned to the target.  If it is not found an error message is generated.  The
routine {\sf error} always yields a null expression.

\subsection{Lists}
\label{listsec}

We have already encountered the behavior of the List data structure
(Figure~\ref{List}) in the discussion of environments.  As with expressions and
environments, lists are represented by a pair of classes.  The first, class {\sf
    ListNode}, maintains the actual list data.  The second, class {\sf List}, is
merely a pointer to a list node, and exists only to provide memory management
operations.

Only one feature of the latter class deserves comment; rather than overloading
the parenthesis operator the class {\sf List} defines a conversion operator
which permits instances of class {\sf List} to be converted without comment to
{\sf ListNodes}.  Thus in most cases a {\sf List} can be used where a {\sf
    ListNode} is expected, and the conversion will be implicitly defined.  We
have seen this already, without having noted the fact, in several places where
the variable {\sf emptyList} (an instance of class {\sf List}) was used in
situations where an instance of class {\sf ListNode} was required.
%
\includecode{list.h}{List}
{The classes {\sf List} and {\sf ListNode}}

The actual list data is maintained in the instance variables {\sf h} and {\sf
    t}, which we have already noted can be retrieved (and, in the case of the h,
set) by the methods {\sf head} and {\sf tail}.  The method {\sf length} returns
the length of a list, and the method {\sf at} permits a list to be indexed as an
array, starting with zero for the head position.

The majority of methods, such as {\sf length}, {\sf at}, {\sf print}, are simple
recursive routines, and will not be discussed.  Only one method is sufficiently
complex to deserve comment, and this is the procedure used to evaluate a list.
A list is interpreted as a function call, and thus the evaluation of a list
involves finding the indicated function and invoking it, passing as arguments
the remainder of the list.  These actions are performed by the method {\sf eval}
shown in Figure~\ref{ListEval}.  An empty list always evaluates to itself.
Otherwise the first argument to the list is examined.  If it is a symbol, a test
is performed to see if it is one of the value-ops.  If it is not found on the
value-op list the first element is evaluated, whether or not it is a symbol.
Generally this will yield a function value.  If so, the method {\sf apply},
which we will discuss in the next section, is used to invoke the function.  If
the first argument did not evaluate to a function an error is indicated.
%
\includecode{list.C}{ListEval}
{The method {\sf eval} from class {\sf List}}

\section{Functions}

If expressions are the heart of the interpreter, then functions are the muscles
that keep the heart working.  All behavior, statements, valueops, as well as
user-defined functions, are implemented as subclasses of class {\sf Function}
(Figure~\ref{Function}).  As we noted in the last section, when a function
(written as a list expression) is evaluated the method {\sf apply}
(Figure~\ref{FunctionApply}) in invoked.  This method takes as argument the
target for the evaluation and a list of unevaluated arguments.  The default
behavior in class {\sf Function} is to evaluate the arguments, using the simple
recursive routine {\sf evalArgs}, then invoke the method {\sf applyWithArgs}.
%
\includecode{function.h}{Function}
{The class {\sf Function}}
%
\includecode{function.C}{FunctionApply}
{The class {\sf Function} method {\sf apply}}

Both the methods {\sf apply} and {\sf applyWithArgs} are declared as virtual,
and can thus be overridden in subclasses.  Those function that do not evaluate
their arguments, such as the functions implementing the control structures of
Chapter 1, override the {\sf apply} method.  Function that do evaluate their
arguments, such as the majority of value-Ops, override the {\sf applyWithArgs}
method.

Two subclasses of {\sf Function} deserve mention.  The class {\sf UnaryFunction}
overrides {\sf apply} to test that only one argument has been provided.
Similarly the class {\sf BinaryFunction} tests for exactly two arguments.  The
remaining major subclass of {\sf Function} is the class {\sf UserFunction}.  We
will defer a discussion of this until we examine the implementation of the {\sf
    define} statement.

\section{The Basic Evaluator}

We are now in a position to finally describe the characteristics that are unique
to the basic evaluator of chapter one.  This interpreter recognizes one command
(the {\sf define} statement), several built-in statements ({\sf if, while, set},
and {\sf begin}), and a number of value-ops.  All are implemented internally as
functions.  What syntactic category a symbol is associated with is determined by
what environment it is placed on, and not by the structure of the function.

\subsection{The define statement}

The define statement is implemented as the single instance of the class {\sf
    DefineStatement} (Figure~\ref{Define}), entered with the key ``define'' in
the {\sf commands} environment.  The class overrides the virtual method {\sf
    apply} (Figure~\ref{DefineApply}), since it must access its arguments before
they are evaluated.  It tests that the arguments are exactly three in number,
and that the first is a symbol and the second a list.  If no errors are
detected, an instance of the class {\sf UserFunction} is created and and set in
the current (always global) environment.
%
\includecode{lisp.h}{Define}
{Declaration of the {\sf define} statement}
%
\includecode{lispPrimitives.C}{DefineApply}
{Implementation of the {\sf define} statement method {\sf apply}}

The class {\sf UserFunction} created by the define statement is similarly a
subclass of class {\sf Function} (Figure~\ref{Userfunction}).  User functions
maintain in instance variables the list of argument names, the body of the
function, and the lexical context in which they are to execute.  These values
are set by the constructor when the function is defined, and freed by the
virtual method {\sf free} when no longer needed.
%
\includecode{function.h}{UserFunction}
{The class {\sf UserFunction}}
%
\includecode{lispPrimitives.C}{UserFunctionApply}
{The class {\sf UserFunction} method of application}

User functions always work with evaluated arguments, and thus they override the
method {\sf applyWithArgs}.  The implementation of this method is shown in
Figure~\ref{UserfunctionApply}.  This method checks that the number of arguments
supplied matches the number in the function definition, then creates a new
environment to match the arguments and their values.  The expression which
represents the body of the function is then evaluated.  By passing the new
context as argument to the evaluation, symbolic references to the arguments will
be matched with the appropriate values.

\subsection{Built-In Statements}

The built-in statements {\sf if, while, set} and {\sf begin} are each defined by
functions entered in the {\sf valueOps} environment.  With the exception of {\sf
    begin}, these must capture their arguments before they are evaluated and
thus, like {\sf define}, they override the method {\sf apply}.

\subsubsection{The If statement}

The If statement (Figure~\ref{Ifstatement}, Figure~\ref{IfstatementApply}) first
insures it has three arguments.  It then evaluates the first argument.  Using
the auxiliary function {\sf isTrue} (Figure~\ref{BasicLispIsTrue}) (which will
vary over the different interpreters as our definition of ``true'' changes) the
truth or falsity of the first expression is determined.  Depending upon the
outcome, either the second or third argument is evaluated to determine the
result.  In the Chapter 1 interpreter the value 0 is false, and all other values
(integer or not) are considered to be true.
%
\includecode{lisp.h}{IfStatement}
{The class {\sf IfStatement}}
%
\includecode{lispPrimitives.C}{IfStatementApply}
{The implementation of the If statement}
%
\includecode{basicLisp.C}{BasicLispIsTrue}
{The implementation of the {\sf isTrue} function}

\subsubsection{The while statement}

The function that implements the while statement is shown in
Figure~\ref{WhileStatementApply}.  Although the while statement requires two
arguments, it nevertheless cannot usefully be made a subclass of class {\sf
    BinaryFunction}, since it must access its arguments before they are
evaluated.  The implementation of the while statements loops until the first
argument evaluates to a true condition, using the same test for true method used
by the if statement.  The results returned by evaluating the body of the while
statement are ignored, as the body is executed just for side effects.
%
\includecode{lispPrimitives.C}{WhileStatementApply}
{The implementation of the While statement}

\subsubsection{The set statement}

The implementation of the set statement is shown in
Figure~\ref{SetStatementApply}.  The function insures the first argument is a
symbol, evaluates the second argument, then sets the binding of the symbol to
value in the current environment.
%
\includecode{lispPrimitives.C}{SetStatementApply}
{The implementation of the set statement}

\subsubsection{The begin statement}

The begin statement evaluate each of its arguments and assigns to the target
variable the value of the last expression (Figure~\ref{BeginStatementApply}).
%
\includecode{lispPrimitives.C}{BeginStatementApply}
{The implementation of the begin statement}

\subsection{The value-Ops}

The Value-ops are functions placed in the {\sf valueop} global environment.
They can be divided into two categories; there are those that take two integer
arguments and produce an integer result ($+$, $-$, $*$, $/$, $=$, $<$ and $>$)
and those that take a single argument ({\sf print}).

The implementation of the integer binary functions is simplified by the
introduction of an intermediate class {\sf IntegerBinaryFunction}, a subclass of
{\sf BinaryFunction} (Figure~\ref{IntegerBinaryFunctionApply}).  The private
state for each instance of this class holds a pointer to a function that takes
two integer values and generates an integer result.  The {\sf applyWithArgs}
method in this class decodes the two integer arguments, then invokes the stored
function to produce the new integer value.  To implement each of the seven
binary integer functions (the relational functions generate 0 and 1 values for
true and false, remember) it is only necessary define an appropriate function
and pass it as argument to the constructor during initialization of the
interpreter.  This can be seen in Figure~\ref{BasicLispInitialize}.

The print function is implemented by a subclass of {\sf UnaryFunction} that
merely invokes the method {\sf print} on the argument.  All expressions will
respond to this method.
%
\includecode{function.h}{IntegerBinaryFunction}
{The class {\sf IntegerBinaryFunction}}
%
\includecode{function.C}{IntegerBinaryFunctionApply}
{Implementation of the {\sf IntegerBinaryFunction} apply}
%
\includecode{lispPrimitives.C}{IntegerArithmeticFunctions}
{Implementation of the arithmetic functions}
%
\includecode{lispPrimitives.C}{IntegerRelationalFunctions}
{Implementation of the relational functions}

\section{Initializing the Run-Time Environment}

Figure~\ref{BasicLispInitialize} shows the initialization routine for the
interpreters of chapter one.  In chapter one there are no global variables
defined at the start of execution.  There is one command, the statement {\sf
    define}, and a number of value-ops.
%
\includecode{basicLisp.C}{BasicLispInitialize}
{Initialization of the Basic Evaluator}

\chapter{The Lisp Interpreter}

The interpreter for Lisp differs only slightly from that of Chapter one.  The
reader/parser is modified so as to recognize quoted constants, two new global
variables ({\sf T} and {\sf nil}) are added, and a number of new value-ops are
defined.  In all other respects it is the same.  Figure~\ref{chap2hier} shows
the class hierarchy for the expression classes added in chapter 2.

\setlength{\unitlength}{5mm}
\begin{figure}
\begin{picture}(16,5)(-4,-3)
\put(-3.5,0){\sf Expression}
\put(0,0.2){\line(1,0){1}}
\put(1,0){\sf Function}
\put(0,0.2){\line(1,-1){1}}
\put(1,-1){\sf QuotedConstant}
\put(4,0.2){\line(1,0){1}}
\put(5,0){\sf BinaryFunction}
\put(4,0.2){\line(1,3){1}}
\put(5,3){\sf UnaryFunction}
\put(9.5,3.2){\line(1,0){1}}
\put(10.5,3){\sf BooleanUnary}
\put(9.5,0.2){\line(1,1){1}}
\put(10.5,1){\sf IntegerBinaryFunction}
\put(9.5,0.2){\line(1,-1){1}}
\put(10.5,-1){\sf BooleanBinaryFunction}
\end{picture}
\caption{Classes added in Chapter Two}\label{chap2hier}
\end{figure}

\section{The Lisp reader}

The Lisp reader is created by subclassing from the base class {\sf Reader}
(Figure~\ref{LispReader}).  The only change is to modify the method {\sf
    readExpression} to check for leading quote marks.  If no mark is found,
execution is as in the default case.  If a quote mark is found, the character
pointer is advanced and the following expression is turned into a quoted
constant.  Note that no checking is performed on this expression.  This permits
symbols, even separators, to be treated as data.  That is, '; is a quoted
symbol, even though the semicolon itself is not a legal symbol.
%
\includecode{lisp.h}{LispReader}
{The Lisp reader/parser}
%
\includecode{lispPrimitives.C}{LispReaderImpl}
{The Lisp reader/parser implementation}

To create quoted constants it is necessary to introduce a new type of
expression.  When an instance of class {\sf QuotedConst} is evaluated, it simply
returns its (unevaluated) data value.

\section{Value-ops}

In addition to adding a number of new value-ops, the Lisp interpreter modifies
the meaning of a few of the Chapter 1 functions.  For example the relational
operators must now return the values {\sf T} or {\sf nil}, and not 1 and 0
values.  Similarly the meaning of {\em true} and {\em false} used by the {\sf
    if} and {\sf while} statements is changed.  Finally the equality testing
function ($=$) must now recognize both symbols and integers.

\subsection{Relationals}

Figure~\ref{EqualFunction} shows the revised definition of the equality testing
function, which now must be prepared to handle symbols and well as integers.
%
\includecode{lispPrimitives.C}{EqualFunction}
{The revised Definition of the equality function}

Implementation of the boolean binary functions is simplified by the introduction
of a class {\sf BooleanBinaryFunction} (Figure~\ref{BooleanBinaryFunction}).
This class decodes the two integer arguments and invokes a further method to
determine the boolean result.  Based on this result either the value of the
global symbol representing true or the symbol representing false is returned.
%
\includecode{function.h}{BooleanBinaryFunction}
{Class returning boolean results from relationals}
%
\includecode{function.C}{BooleanBinaryFunctionApply}
{{\sf BooleanBinaryFunction} apply function}

Finally Figure~\ref{LispIsTrue} shows the revised function used by {\sf if} and
{\sf while} statements to determine the truth or falsity of their condition.
Unlike in Chapter 1, where 0 and 1 were used to represent true and false, here
{\sf nil} is used as the only false value.
%
\includecode{lisp.C}{LispIsTrue}
{Specialised {\sf isTrue} for lisp}

\subsection{Car, Cdr and Cons}

{\sf car} and {\sf cdr} are implemented as simple unary functions
(Figure~\ref{CarCdrCons}), and {\sf cons} is a simple binary function that
creates a new {\sf ListNode} out of its two arguments.
%
\includecode{lispPrimitives.C}{CarCdrCons}
{{Implementation of {\sf car}, {\sf cdr} and {\sf cons}}}

\footnote{A matter for debate is whether Cons should give an error if the second
    argument is not a list.  Real Lisp doesn't care; but also uses a different
    format for printing such lists.  Our interpreter prints such as lists
    exactly as if the second argument had been a list containing the element.}

\subsection{Predicates}

The implementation of the predicates {\sf number?}, {\sf symbol?}, {\sf list?}
and {\sf null?} is simplified by the creation of a class {\sf BooleanUnary}
(Figure~\ref{BooleanUnary}), subclassing {\sf UnaryFunction}.  As with the
integer functions implemented in chapter 1, instances of {\sf BooleanUnary}
maintain as part of their state a function that takes an expression and returns
an integer (that is, boolean) value.  Thus for each predicate it is only
necessary to write a function which takes the single argument and returns a
true/false indication.
%
\includecode{lisp.h}{BooleanUnary}
{The class {\sf BooleanUnary}}
%
\includecode{lispPrimitives.C}{BooleanUnaryApply}
{{\sf BooleanUnary} apply function}

\section{Initialization of the Lisp Interpreter}

Figure~\ref{LispInitialize} shows the initialization method for the Lisp
interpreter.
%
\includecode{lisp.C}{LispInitialize}
{Initialization of the Lisp interpreter}

\chapter{The APL Interpreter}

My version of the APL interpreter differs somewhat from that provided by Kamin:
\begin{itemize}
    \item My version will recognize arbitrary rank (dimension) arrays, not
    simply scalar, vector and two dimensional arrays.  (Although currently it is
    only able to print those three types).
    \item The C++ version of the interpreter recognizes vector constants without
    the necessity for quoting them, as in (resize (3 4) (indx 12)).
    \item I have eliminated the if and while statements, thus forcing
    programmers into a more ``APL'' style of thought.
    \item My version of catenation works now for values of arbitrary
    dimensionality.  (Transpose and print are the only two functions that limit
    the dimensionality of their arguments).
\end{itemize}

Despite the APL interpreter being larger than any other interpreter, I think
that the addition of a few more functions could give the student an even better
feel for the language, as well as providing a smooth transition to functional
programming.  Specifically, I think reduction should be defined as a functional,
and inner and outer products added as operations.  I have not done this as yet,
however.

Figure~\ref{aplhier} shows the class hierarchy for the classes introduced in
this chapter.

\setlength{\unitlength}{5mm}
\begin{figure}
    \begin{picture}(25,10)(-4,-5)
        \put(-3.5,0){\sf Expression}
        \put(0,0.2){\line(1,0){1}}
        \put(1,0){\sf Function}
        \put(0,0.2){\line(1,-2){1}}
        \put(1,-2){\sf APLValue}
        \put(4,0.2){\line(1,-2){1}}
        \put(5,-2){\sf BinaryFunction}
        \put(9.5,-1.8){\line(1,0){1}}
        \put(10.5,-2){\sf APLBinaryFunction}
        \put(4,0.2){\line(1,2){1}}
        \put(5,2){\sf UnaryFunction}
        \put(9.5,2.2){\line(1,0){1}}
        \put(10.5,2){\sf APLUnaryFunction}
        \put(16,2.2){\line(1,0){1}}
        \put(17,2){\sf RavelFunction}
        \put(16,2.2){\line(1,1){1}}
        \put(17,3){\sf ShapeFunction}
        \put(16,2.2){\line(1,2){1}}
        \put(17,4){\sf APLReduction}
        \put(16,2.2){\line(1,-1){1}}
        \put(17,1){\sf IndexFunction}
        \put(16,2.2){\line(1,3){1}}
        \put(17,5){\sf TransposeFunction}
        \put(16,-1.8){\line(1,0){1}}
        \put(17,-2){\sf RestructFunction}
        \put(16,-1.8){\line(1,1){1}}
        \put(17,-1){\sf CompressFunction}
        \put(16,-1.8){\line(1,-1){1}}
        \put(17,-3){\sf APLScalarFunction}
        \put(16,-1.8){\line(1,-2){1}}
        \put(17,-4){\sf CatenationFunction}
        \put(16,-1.8){\line(1,-3){1}}
        \put(17,-5){\sf SubscriptionFunction}
    \end{picture}
    \caption{The APL interpreter class hierarchy}\label{aplhier}
\end{figure}

\section{APL Values}

The APL interpreter manipulates APL values, which are defined by the data type
{\sf APLValue} (Figure~\ref{APLValue}).  An APL value represents a integer
rectilinear array.  Internally, such a value is represented by a list that
maintains the shape (extent along each dimension) and a vector of integer
values.  The length of the shape list provides the rank (dimensionality) of the
data value.  The product of the values in the shape indicates the number of
elements in the array, except in the case of scalar values, which have an empty
shape array.
%
\includecode{apl.C}{APLValue}
{The Representation for APL Values}

APL values are stored in what is called {\em ravel-order}.  This is what in some
other languages is called row-major order.

The methods defined for APL values can be used to determine the number of
elements contained in the structure ({\sf size}), obtain the shape of the value
({\sf shape}), obtain the shape at any given dimension ({\sf shapeAt}), obtain
the value at any given ravel-order position ({\sf at}), and finally change the
value at any position ({\sf atPut}).

\section{The APL Reader}

The APL reader is modified so that individual scalar values and vectors of
integers are recognized as APL values.  The class definition for {\sf APLreader}
is shown in Figure~\ref{APLreader}, and the code for the two auxiliary functions
in the next figure.
%
\includecode{apl.C}{APLreader}
{The APL reader}
%
\includecode{apl.C}{readAPLscalar}
{The APL reader functions}

\section{APL Functions}

The implementation of the APL functions is simplified by the addition of two
auxiliary classes, {\sf APLUnary} and {\sf APLBinary}.  In addition to checking
that the right number of arguments are provided to a function application, these
check to insure that the arguments are APL values \footnote{A largely gratuitous
    move, since the user has no way of creating anything other than an APL
    value.  Still, it doesn't do any harm to be careful.} and invoke yet another
virtual function {\sf applyOp}, to perform the actual calculation.

\subsection{Scalar Functions}

By far the largest class of APL functions are the so-called {\em scalar
    functions}.  These are the conventional arithmetic and logical functions,
such as addition and multiplication, extended in the natural way to arrays.  The
only complication in the implementation of these values concerns what is called
{\em scalar extension}.  That is, a scalar value can be used as either the left
or right argument to a scalar function, and it is treated as if it were an
entire array of the correct dimensionality to match the other argument.  Since
scalar extension can occur with either the left or right argument, the code for
scalar functions divides naturally into three cases.

Scalar functions are implemented using a single class by making use, as we have
done before, of an instance variable that contains a pointer to a integer
function that generates an integer result.  The class {\sf APLscalarFunction}
and the method {\sf applyOp} are shown in Figure~\ref{APLScalarFunctionApply}.
Note that the same functions used in the previous interpreters can be used in
the construction of the APL scalar functions.
%
\includecode{apl.C}{APLScalarFunctionApply}
{APL Scalar Functions}

\subsection{Reduction}

For each scalar function there is an associated reduction function.\footnote{The
    statement is true of real APL.  The Kamin interpreters do not implement
    reductions with relational operators, which are, however, not particularly
    useful.}  Reduction in these interpreters always occurs along the last
dimension.  Thus to compute the size of a new value is suffices to remove the
last dimension value.  This also simplifies the generation of the new values,
since the argument array can be processed in units as long as the final
dimension.  As with the scalar functions, there is one class defined for all the
reductions, with each instance of this class maintaining the particular scalar
function being used for the reduction operations.  Figure~\ref{APLReduction}
shows the code used in computing the APL reduction function.
%
\includecode{apl.C}{APLReduction}
{Implementation of the APL reduction function}

\subsection{Compression}

Compression, like reduction, operates on the last dimension of a higher order
array, changing its extent to that of the number of one elements in the
left-argument vector.  The length of the left argument vector must match the
extent of the last dimension of the right argument.  The compression function
(Figure~\ref{APLCompressionFunctionApply}) first computes the number of one
elements in the left argument, then iterates over the right argument generating
the new values.
%
\includecode{apl.C}{APLCompressionFunctionApply}
{The Compression function}

\subsection{Shape and Reshape}

The {\sf shape} function merely copies the size on its argument into a new APL
value.  The reshape function ({\sf restruct}) generates a new value with a size
given by the left argument, which must be a vector, using elements from the
right argument, recycling over the ravel ordering of the right argument multiple
times if necessary.  The implementation of these functions is shown in
Figure~\ref{APLShapeFunctionApply} and ref{APLRestructFunctionApply}.
%
\includecode{apl.C}{APLShapeFunctionApply}
{The shape function}
%
\includecode{apl.C}{APLRestructFunctionApply}
{The reshape function}

\subsection{Ravel and Index}

The ravel function (Figure~\ref{APLRavelFunctionApply}) merely takes an argument
of arbitrary dimensionality and returns the values as a vector.  The index
function (called iota in real APL) (Figure~\ref{APLIndexFunctionApply}) takes a
scalar value and returns a vector of numbers from 1 to the argument value.
%
\includecode{apl.C}{APLRavelFunctionApply}
{Ravel}
%
\includecode{apl.C}{APLIndexFunctionApply}
{Index}

\subsection{Catenation}

The catenation function joins two arrays along their last dimension.  They must
match in all other dimensions.  To build the new result first a row from the
first array is copies into the final array, then a row from the second array,
then another row from the first, followed by another row from the second, and so
on until all rows from each argument have been used.
%
\includecode{apl.C}{APLCatenationFunctionApply}
{Implementation of the Catenation function}

\subsection{Transpose}

While real APL defines transpose for arbitrary dimension arrays, the transpose
presented here works only for arrays of dimension two or less.  For vector and
scalars the transpose does nothing.  Thus the only code required
(Figure~\ref{APLTransposeFunctionApply}) is to take the transpose of a two
dimensional array.
%
\includecode{apl.C}{APLTransposeFunctionApply}
{The Transpose Function}

\subsection{Subscription}

The Pascal interpreter provided by Kamin applies subscription to the first
dimension of a multidimension value.  In order to be consistent with the other
functions, my version does subscription along the last dimension.  Neither is
exactly the same as the real APL version.  The subscription code is shown in
Figure~\ref{APLSubscriptFunction}.
%
\includecode{apl.C}{APLSubscriptFunction}
{The Subscription function}

\section{Initialization of the APL interpreter}

The initialization code for the APL interpreter is shown in
Figure~\ref{APLInitialize}.
%
\includecode{apl.C}{APLInitialize}
{APL interpreter initialization}

\chapter{The Scheme Interpreter}

After all the code required to generate the APL interpreter of Chapter 3, the
Scheme interpreter is simplicity in itself.  Of course, this has more to do with
the similarity of Scheme to the basic Lisp interpreter of Chapter 2 than with
any differences between APL and Scheme.

To implement Scheme it is only necessary to provide an implementation of the
lambda function.  This is accomplished by the class {\sf Lambda}, shown in
Figure~\ref{SchemeLambdaFunction}.  The actual implementation of lambda uses the
same class {\sf UserFunction} we have seen in previous chapters.
%
\includecode{scheme.C}{SchemeLambdaFunction}
{The class Lambda}

Initialization of the Scheme interpreter differs slightly from the code used to
initialize the Lisp interpreter (Figure~\ref{SchemeInitialize}).  The {\sf
    define} command is no longer recognized, having been replaced by the {\sf
    set}/{\sf lambda} pair.  The built-in arithmetic functions are now considred
to be global symbols, and not value-ops.  Indeed, there are no comands or
value-ops in this language.
%
\includecode{scheme.C}{SchemeInitialize}
{Initialization of the Scheme Interpreter}

\chapter{The SASL interpreter}\label{sasl}

The SASL interpreter is largely constructed by removing features from the Scheme
interpreter, such as while loops and so on, and changing the implementation of
the {\sf cons} function to add delayed evaluation.  Figure~\ref{saslhier} shows
the class hierarchy for the classes added in this chapter.

\setlength{\unitlength}{5mm}
\begin{figure}
    \begin{picture}(25,10)(-4,-5)
        \put(-3.5,0){\sf Expression}
        \put(0,0.2){\line(1,0){1}}
        \put(1,0){\sf Function}
        \put(0,0.2){\line(1,-2){1}}
        \put(1,-2){\sf Thunk}
        \put(4,0.2){\line(1,-2){1}}
        \put(5,-2){\sf UserFunction}
        \put(9.5,-1.8){\line(1,0){1}}
        \put(10.5,-2){\sf LazyFunction}
        \put(4,0.2){\line(1,2){1}}
        \put(5,2){\sf SaslConsFunction}
        \put(4,0.2){\line(1,0){1}}
        \put(5,0){\sf LambdaFunction}
    \end{picture}
    \caption{Class Hierarchy for expressions in the SASL interpreter}
    \label{saslhier}
\end{figure}

\subsection{Thunks}

Delayed evaluation is provided by adding a new expression type, called the {\em
    thunk}.  Figure~\ref{SASLThunk} shows the data structure used to represent
this type of value.  Every thunk maintains a boolean value indicating whether
the thunk has been evaluated yet, an expression (representing either the
unevaluated or evaluated expression, depending upon the state of the boolean
flag), and a context in which the expression is to be evaluated.  Thunks print
either as three dots, if they have not yet been evaluated, or as the printed
representation of their value, if they have.
%
\includecode{sasl.C}{SASLThunk}
{Definition of Thunks}
%
\includecode{sasl.C}{SASLThunkTouch}
{Definition of Thunk touch}

Here we finally see an overridden definition for the method {\em touch}.  You
will recall that this method was defined in Chapter 1, and that all other
expressions merely return their value as the result of this expression.  Thunks,
on the other hand, will evaluate themselves if touched, and then return their
new evaluated result.  With the addition of this feature many of the definitions
we have presented in earlier chapters, such as the definitions of {\sf car} and
{\sf cdr}, hold equally well when given thunks as arguments.

Since thunks can represent lists, symbols, integers and so on, the predicate
methods {\sf isSymbol} and the like must be redefined as well.  If the thunk
represents an evaluated value, these simply return the result of testing that
value (Figure~\ref{SASLThunkPredicates}).
%
\includecode{sasl.C}{SASLThunkPredicates}
{Thunk predicates}
%
\includecode{sasl.C}{SASLThunkEval}
{Thunk eval}

\section{Lazy Cons}

The SASL cons function differs from the Scheme version in producing a list node
containing a pair of thunks, rather than a pair of values
(Figure~\ref{SaslConsFunction}).  Class {\sf SaslConsFunction} must now be a
subclass of {\sf Function} and not {\sf BinaryFunction}, because it must grab
its arguments before they are evaluated.  Thus it must itself check to see that
only two arguments are passed to the function.
%
\includecode{sasl.C}{SaslConsFunction}
{The Sasl Lazy Cons function}

\section{Lazy User Functions}

User defined functions must be provided with lazy evaluation semantics as well.
This is accomplished by defining a new class {\sf LazyFunction}
(Figure~\ref{SASLLazyFunction}).  Lazy functions act just like user functions
from previous chapters, only they do not evaluate their arguments.  Thus the
function body is evaluated by the method {\sf apply}, rather than passing the
evaluated arguments on to the method {\sf applyWithArgs}.  The lambda function
from the previous chapter is modified to produce an instance of {\sf
    LazyFunction}, rather than {\sf UserFunction}.
%
\includecode{sasl.C}{SASLLazyFunction}
{The implementation of lazy functions}

\chapter{The CLU interpreter}

The CLU interpreter is created by introducing a new datatype, the cluster,
and three new types of functions.  Constructors create new instances of a
cluster, selectors access a portion of a cluster state, and modifiers change
a portion of a cluster state.  Figure~\ref{cluhier} shows the class
hierarchy for the classes added in this chapter.

\setlength{\unitlength}{5mm}
\begin{figure}
    \begin{picture}(25,10)(-4,-5)
        \put(-3.5,0){\sf Expression}
        \put(0,0.2){\line(1,0){1}}
        \put(1,0){\sf Function}
        \put(0,0.2){\line(1,-2){1}}
        \put(1,-2){\sf Cluster}
        \put(4,0.2){\line(1,-2){1}}
        \put(5,-2){\sf BinaryFunction}
        \put(9.5,-1.8){\line(1,0){1}}
        \put(10.5,-2){\sf Modifier}
        \put(4,0.2){\line(1,2){1}}
        \put(5,2){\sf UnaryFunction}
        \put(9.5,2.2){\line(1,0){1}}
        \put(10.5,2){\sf Selector}
        \put(4,0.2){\line(1,0){1}}
        \put(5,0){\sf Constructor}
        \put(4,0.2){\line(1,4){1}}
        \put(5,4){\sf ClusterDef}
    \end{picture}
    \caption{Class Hierarchy for the CLU interpreter}
    \label{cluhier}
\end{figure}

\section{Clusters}

A cluster simply encapsulates a series of names and values, hiding them from
normal examination.  The most natural way to do this is for a cluster to
maintain an environment (Figure~\ref{CLUCluster}).  The predicate {\sf
    isCluster} returns this environment value.
%
\includecode{clu.C}{CLUCluster}
{The definition of a cluster value}

To create a cluster requires a constructor function.  The constructor is
provided with a list of names of the elements in the internal representation of
the cluster, and simply insures that the arguments it is provided with match in
number of the names it maintains.

\section{Selectors and Modifiers}

To access or modify the elements of a constructor requires functions called
selectors or modifiers.  Each of these maintain as their state the name of the
field they are responsible for.  When invoked with a constructor, the access or
change their given field.
%
\includecode{clu.C}{CLUSelectorModifier}
{Selectors and Modifiers for clusters}

\section{Defining clusters}

It thus remains only to give the (rather lengthy) definition of the function
that generates constructor information from the textual description.  (We do not
say generates clusters themselves, for that is the responsibility of the
constructor functions).  This function is shown in Figure~\ref{CLUClusterDef}.
It rips apart a cluster definition and does the right things (need a better
description here, but I don't have time to write it now).  (Need to point out
that cluster functions have an internal and an external name, and these are put
of different environments).  (I suppose an alternative would have been to
introduce a new datatype for two part names, which when evaluated would look up
their second part in the cluster provided by their first part).
%
\includecode{clu.C}{CLUClusterDef}
{The cluster recognition function}

\chapter{The Smalltalk interpreter}

As with chapter 3, with the Smalltalk interpreter I have also made a number of
changes.  These include the following:

\begin{itemize}
    \item I have changed the syntax for message passing.  The first argument in
    a message passing expression is an object, which is defined (for
    implementation purposes) as a type of function.  The second argument must be
    the message selector, a symbol.  This change is not only produces a syntax
    that is slightly more Smalltalk-like, but it more closely reinforces the
    critical object-oriented idea that the interpretation of a message depends
    upon the receiver for that message.
    \item Integers are objects, and respond to messages.  The most obvious
    effect of this is to restore infix syntax for arithmetic operations, since
    (3 + 4) is interpreted (Smalltalk-like) as the message ``+'' being passed to
    the object 3 with argument 4.
    \item The initial environment is very spare.  There are only the two classes
    {\sf Object} and {\sf Integer}, which respond to the messages {\sf
        subclass}, {\sf method} and {\sf new}, and integer instances that
    respond to arithmetic messages.
    \item The {\sf if} command is a message sent to integers (0 for false and
    non-zero for true).  This is also more Smalltalk-like.  The following
    expression sets {\sf z} to the minimum of {\sf x} and {\sf y}.
    \begin{center} {\sf ((x $<$ y) if (set z x) (set z y))}
    \end{center}
    \item The only non-message statements are the assignment statement {\sf set}
    and the {\sf begin} statement.  (Note - there is no loop.  I couldn't think
    of a good way to do this within the syntax given using message passing (no
    blocks!) but I don't think this will be too great a problem; recursion can
    be used in most cases where looping is used currently).
\end{itemize}

A class hierarchy for the classes added in this chapter is shown in
Figure~\ref{smallhier}.

\setlength{\unitlength}{5mm}
\begin{figure}
    \begin{picture}(25,10)(-4,-5)
        \put(-3.5,0){\sf Expression}
        \put(0,0.2){\line(1,0){1}}
        \put(1,0){\sf Function}
        \put(0,0.2){\line(1,4){1}}
        \put(1,4){\sf Symbol}
        \put(3,4.2){\line(1,0){1}}
        \put(4,4){\sf SmalltalkSymbol}
        \put(4,0.2){\line(1,-2){1}}
        \put(5,-2){\sf UserFunction}
        \put(9,-1.8){\line(1,0){1}}
        \put(10,-2){\sf method}
        \put(13,-1.8){\line(1,0){1}}
        \put(14,-2){\sf SubclassMethod}
        \put(13,-1.8){\line(1,1){1}}
        \put(14,-1){\sf NewMethod}
        \put(13,-1.8){\line(1,2){1}}
        \put(14,0){\sf IntegerBinaryMethod}
        \put(13,-1.8){\line(1,-1){1}}
        \put(14,-3){\sf IfMethod}
        \put(13,-1.8){\line(1,-2){1}}
        \put(14,-4){\sf MethodMethod}
        \put(4,0.2){\line(1,2){1}}
        \put(5,2){\sf Object}
        \put(7,2.2){\line(1,0){1}}
        \put(8,2){\sf IntegerObject}
    \end{picture}
    \caption{Expression class hierarchy for Smalltalk interpreter}
    \label{smallhier}
\end{figure}

\section{Objects and Methods}

An object is an encapsulation of behavior and state.  That is, an object
maintains, like a cluster, certain state information accessible only within the
object.  Similarly objects maintain a collection of functions, called {\em
    methods}, that can be invoked only via message passing.  Internally, both
these are represented by environments (Figure~\ref{SmalltalkObject}).  The
methods environment contains a collection of functions, and the data environment
contains a collection of internal variables.  Objects are declared as a subclass
of {\sf function} so that normal function syntax can be used for message
passing.  That is, a message is written as

\begin{center}
{\sf (object message arguments)}
\end{center}
%
\includecode{smalltalk.C}{SmalltalkObject}
{Classes for Object and Method}

Methods are similar to conventional functions (and are thus subclasses of {\sf
    UserFunction}) in that they have an argument list and body.  Unlike
conventional functions they have a receiver (which must always be an object) and
the environment in which the method was created, as well as the environment in
which the method is invoked.  Thus methods define a new message {\sf doMethod}
that takes these additional arguments.

A subtle point to note is that the creation environment in normal functions is
captured when the function is defined.  For objects this environment cannot be
defined when the methods are created, but must wait until a new instance is
created.  Our implementation waits even longer, and passes it as part of the
message passing protocol.

The mechanism of message passing is defined by the function {\sf apply} in class
{\sf Object} (Figure~\ref{SmalltalkObjectApply}).  Messages require a symbol for
the first argument, which must match a method for the object.  This method is
then invoked.  Similarly Figure~\ref{SmalltalkObjectApply} shows the execution
of normal methods (that is, those methods other than the ones provided by the
system).  The execution context is set for the method, and the receiver is added
as an implicit first argument, called {\sf self} in every method.  The method is
then invoked as if it were a conventional function.
%
\includecode{smalltalk.C}{SmalltalkObjectApply}
{Implementation of Message Passing}

\section{Classes}

Classes are simply objects.  As such, they respond to certain messages.  In our
Smalltalk interpreter there are initially two classes, {\sf Object} and {\sf
    Integer}.  The class {\sf Object} is a superclass of {\sf Integer}, and is
typically the superclass of most user defined classes as well.  There are
initially three messages that classes respond to:

\begin{itemize}
    \item {\sf subclass}.  This message is used to create new classes, as
    subclasses of existing classes.  Any arguments provided are treated as the
    names of instance variables (local state) to be generated when instances of
    the new classes are created.  The new class is returned as an object, and is
    usually immediately assigned to a global variable.  The syntax for new
    classes is thus similar to the following:
    \begin{center}
        {\sf (set Foo (Object subclass x y z))}
    \end{center}
    which creates a new class with three instance variables, and assigns this
    class to the variable {\sf Foo}.  Subclasses can also access instance
    variables defined in classes.

    It is legal to subclass from class {\sf Integer}, although the results are
    not useful for any purpose.

    \item {\sf new}.  This message, which takes no arguments, is used to create
    a new instance of the receiver class.  The new instance is returned as the
    result of the method, as in the following:
    \begin{center}
        {\sf (set newfoo (Foo new))}
    \end{center}
    Although the class {\sf Integer} responds to the message {\sf new}, no
    useful value is returned.  (Real Smalltalk has something called {\em
        metaclasses} that can be used to prevent certain classes from responding
    to all messages.  Our Smalltalk doesn't).
    \item {\sf method}.  This message is used to define a new method for a
    class.  Following the keyword {\sf method} the syantx is the same as a
    normal function definition.  Within a method the pseudo-variable {\sf self}
    can be used to represent the receiver for the method.
    \begin{center}
        {\sf (Integer method square () (self * self))}
    \end{center}
\end{itemize}

Classes are represented in the same format as other objects.  They act as if
they held two instance variables; {\sf names}, which contains a list of instance
variable names for the class, and {\sf methods}, which contains the table of
method definitions for the class.  Note that these are held in the data area for
the class.  (A picture might help here...).

The implementation of the method {\sf subclass} is shown in
Figure~\ref{SmalltalkSubclassMethod}.  The instance variables for the parent
class is obtained, and the new instance variables for the class added to them.
Inheritance is implemented by creating a new empty method table, but having it
point to the method table for the parent class.  Thus a search of the method
table for the newly created class will automatically search the parent class if
no overriding method is found.  These two values are inserted as data values in
the new class object.  The methods a class responds to will be exactly the same
as those of the parent class (thus all classes respond to the same messages).
%
\includecode{smalltalk.C}{SmalltalkSubclassMethod}
{Implementation of the {\sf subclass} method}
%
\includecode{smalltalk.C}{SmalltalkObjectGetNames}
{Implementation of Object getNames}

The implementation of the method {\sf new}, shown in
Figure~\ref{SmalltalkNewMethodDoMethod}, gets the list of instance variables
associated with the class.  A new environment is then created that assigns an
empty value to each variable.  Using the method table stored in the data area
for the class object a new object is then created.
%
\includecode{smalltalk.C}{SmalltalkNewMethodDoMethod}
{The method {\sf new}}
%
\includecode{smalltalk.C}{SmalltalkMethodMethodDoMethod}
{The method {\sf method}}

The method used to respond to the {\sf method} command is shown in
Figure~\ref{SmalltalkMethodMethodDoMethod}.  This is very similar to the
function used to break apart the {\sf define} command in Chapter 1.  The only
significant difference includes the addition of the receiver {\sf self} as an
implicit first parameter in the argument list, and the fact that the function is
placed in a method table, rather than in the global environment.

\section{Symbols and Integers}

Symbols in Smalltalk have no property other than they evaluate to themselves,
and are guaranteed unique.  They are easily implemented by subclassing the
existing class {\sf Symbol} (Figure~\ref{SmalltalkSymbol}), and modifying the
reader/parser to recognize the tokens.  (Unlike symbols in real Smalltalk, our
symbols are not objects and will not respond to any messages).
%
\includecode{smalltalk.C}{SmalltalkSymbol}
{Symbols in Smalltalk}
%
\includecode{smalltalk.C}{SmalltalkInteger}
{Integers in Smalltalk}

Integers are also redefined as objects, and a built-in method {\sf
    IntegerBinaryMethod} (Figure~\ref{SmalltalkInteger}), similarly to {\sf
    IntegerBinaryFunction}, is created to simplify the arithmetic methods.

Control flow is implemented as a message to integers.  (In real Smalltalk
control flow is implemented as messages, but to different objects).  If the
receiver is zero the first argument to the if method is returned, otherwise the
second argument is returned.
%
\includecode{smalltalk.C}{SmalltalkIfMethod}
{Implementation of the if method}

\section{Smalltalk reader}

The Smalltalk reader subclasses the reader class so as to recognize
integers and symbols (Figure~\ref{SmalltalkReader}).
%
\includecode{smalltalk.C}{SmalltalkReader}
{The Smalltalk reader}

\section{The big bang}

To initialize the interpreter we must create the objects {\sf Object} and {\sf
    Integer}.  (Need more explanation here, but I'll just give the code for
now).
%
\includecode{smalltalk.C}{SmalltalkInitialize}
{Initializing the Smalltalk interpreter}

\chapter{The Prolog interpreter}

As with chapters 3 and 7, I have in this chapter taken great liberties with the
syntax used by Kamin in his interpreter.  However, unlike chapters 3 and 7,
where my intent was to make the interpreters closer in spirit to the original
language, my intent here is to simplify the interpreter.  Specifically, I wanted
to build on the base interpreter, just as we have done for all other languages.
I am able to do this by adopting {\em continuations} as the fundamental basis
for my implementation, and by basing the code on slightly different primitives.

The language used by this interpreter has the following characteristics:
\begin{itemize}
    \item As in real prolog, the only basic objects are symbols.  I've even
    tossed out integers, just to simplify things.  Symbols have no meaning other
    than their uniqueness.  Those symbols beginning with lower case letters are
    atomic, while those beginning with upper case letters are variables.
    \item There are two basic statement types, the {\sf define} statement we
    have seen all through the interpreters, and a new statement called {\sf
        query}.  The later is used to form questions.
    \item The bodies of functions or queries can be composed of four types of
    relations:
    \begin{itemize}
        \item {\sf (print x)} which if x is defined prints the value of x and is
        successful, and if x is not defined is not successful.
        \item {\sf (:=: x y)} which attempts to unify x and y, which can be
        either variables or symbols.  The order of arguments is unimportant.
        \item {\sf (and $rel_1$ $rel_2$ ... )} which can take any number of
        relational arguments and is successful if all the relations are
        successful.  Relations are tried in order.
        \item {\sf (or $rel_1$ $rel_2$ ... )} which can take any number of
        relational arguments and is successful if one one of the relations is
        successful.  They are tried in order.
    \end{itemize}
\end{itemize}

For example, suppose {\sf sam} is the father of {\sf alice}, and {\sf alice} is
the mother of {\sf sally}.  We might encode this in a parent database as
follows:

\begin{lstlisting}
-> (define parent (X Y)
       (or
           (and (:=: X alice) (:=: Y sally))
               (and (:=: X sam) (:=: Y alice))
       ))
\end{lstlisting}

The query statement can then be used to ask queries of the database.  For
example, we can find out who is the parent of {\sf alice} as follows:

\begin{lstlisting}
-> (query (and (parent X alice) (print X)))
sam
ok
\end{lstlisting}

Or we can find the child of {\sf alice} with the following:

\begin{lstlisting}
-> (query (and (parent alice X) (print X)))
sally
ok
\end{lstlisting}

If we ask a question that does not have an answer, the response not-ok is
printed.

\begin{lstlisting}
-> (query (and (parent fred X) (print X)))
not ok
\end{lstlisting}

Prolog style rules can be introduced using the same form we have been using
for functions.

\begin{lstlisting}
-> (define grandparent (X Y)
        (and (parent X Z) (parent Z Y)))
-> (query (and (grandparent A B) (print A) (print B)))
sam
sally
ok
\end{lstlisting}

There is no built-in way to force a relation to cycle through all
alternatives.  However, this is easily accomplished by making a relation
that will always fail, for example trying to unify apples with oranges:

\begin{lstlisting}
-> (define fail () (:=: apples oranges))
\end{lstlisting}

We can then use this to print out all the parents in our database.
Notice that not-ok is printed, since we eventually fail.

\begin{lstlisting}
-> (query (and (parent X Y) (print X) (fail)))
sam
alice
not ok
\end{lstlisting}

Note - although it might appear the use of and's and or's is more powerful than
writing rules in horn clauses, in fact they are identical; although horn clauses
will often require the introduction of unnecessary names.  I myself find this
formulation more natural, although I'm not exactly unbiased.

Unification of two unknown symbols works as expected.  If any symbol
subsequently becomes defined, the other is defined as well.
\begin{lstlisting}
-> (define same (X Y) (:=: X Y))
-> (query (and (same A B) (:=: A sally) (print B)))
sally
ok
\end{lstlisting}

I will divide the discussion of the implementation into three parts.  These are
unification, symbol management, and backtracking.

\section{Unification}

Unification is the basis for logic programming.  Using unification, unbound
variables can be bound together.  As we saw in the last example, this is more
than simple assignment.  If two unknown variables are unified together and
subsequently one is bound, the other should be bound also.  Unification also
differs from assignment in that it can be ``undone'' during the process of
backtracking.

Unification is most easily implemented by introducing a level of indirection.
Prolog values will be represented by a new type of expression, called {\sf
    PrologValue} (Figure~\ref{PrologValue}).  Instances of this class maintain a
data value, which is either undefined (that is, null), a symbol, or another
prolog value.  The prolog reader is modified so as to return a prolog value were
formerly a symbol was returned.  (Also the reader will no longer recognize
integers, which are not used in our simplified interpreter).

A prolog value that contains a symbol is used to represent the prolog symbol of
the same name.  A prolog value that contains an empty data value represents a
currently unbound value.  Finally a prolog value that points to another prolog
value represents the unification of the first value with the second.  Whenever
we need the value of a prolog symbol, we first run down the chain of
indirections to get to the bottom of the sequence.  (This is done automatically
by the overridden method {\sf isSymbol}, which will yield the symbol value
behind arbitrary levels of indirection if a prolog value represents a symbol.)
%
\includecode{prolog.C}{PrologValue}
{The class declaration for prolog values}

The unification algorithm is shown in Figure~\ref{PrologUnify}.  For reasons we
will return to when we discuss backtracking, the algorithm takes three
arguments.  The first is a reference to a pointer to a prolog value.  If the
unification process changes the value of either the the two other arguments, the
pointer in the first argument is set to the altered value.
%
\includecode{prolog.C}{PrologUnify}
{The Unification process}

The unification process divides naturally into three parts.  If either argument
is undefined, it is changed so as to point to the other arguments.  This is true
regardless of the state of the other argument.  This is how two undefined
variables can be unified - the first is set to point to the second.  If the
second is subsequently changed, the first will still indirectly point to the new
value.  Suppose it is, however, the first that is subsequently changed?  In that
case the next portion of the unification algorithm is entered.  If both
arguments are defined and either one is an indirection, then we simply try to
unify the next level down in the pointer chains.  (Note: Lots of pictures would
make this clearer, but I don't have time right now..).  If neither argument is
undefined nor an indirection, they both must be symbols.  In that case,
unification is successful if and only if they have the same textual
representation.

\section{Symbol Management}

The only significant problem here is that symbolic constants must evaluate to
themselves and that symbolic variables can be introduced without declaration.
We see the latter in sequences such as:

\begin{lstlisting}
-> (define grandparent (X Y)
        (and (parent X Z) (parent Z Y)))
-> (query (and (grandparent A B) (print A) (print B)))
sam
sally
ok
\end{lstlisting}

Here the variable Z suddenly appears without prior use.  The solution to both of
these problems is found in the code used to respond to the {\sf eval} request
for a Prolog value.  This code is shown in Figure~\ref{PrologEval}.  The virtual
method {\sf isSymbol} runs down any indirection links, returning the symbol data
value if the last value in a chain of indirections represents a symbolic
constant.  If a symbol is found, we first look to see if the symbol is bound in
the current environment.  If so we simply return its binding\footnote{Checking
    for bindings before checking the first letter allows rules to begin with
    lower case letters, which seems to be more natural to most programmers.}.
If not, if the symbol begins with a lower case letter it evaluates to itself,
and so we simply return it.  If it is not a symbolic constant, than it is a new
symbolic variable, and we add a binding to the current environment to indicate
that the value is so-far undefined.  Thus new symbols are added to the current
environment as they are encountered, instead of generating error messages as
they did in previous interpreters.
%
\includecode{prolog.C}{PrologEval}
{Evaluation of a prolog symbol}

\section{Backtracking}

The seem to be two general approaches to implementing logic programming
languages.  The technique used by most modern prolog systems is called the WAM,
or Warren Abstract Machine.  The WAM performs backtracking by not popping the
activation frame stack when a procedure is terminated, and saving enough
information to restart the procedure in a record called the ``choice point''.
Since in our interpreters calling a function is performed by recursively calling
evaluation routines inside the interpreter, the activation stack for the users
program is held in part in the activation stack for the interpreter itself.
Thus it is difficult for us to manipulate the activation record stack directly.
The alternative technique, which is actually historically older, is to build up
an unevaluated expression that represents what it is you want to do next before
you ever start execution.  This is called a continuation, and we were introduced
to this idea in the chapter on Scheme.  When we are faced with a choice, we can
then try one alternative and the continuation, and if that doesn't work try the
next.

In general continuations are simply arbitrary expressions representing ``what to
do next''.  In our case they will always return a boolean value, indicating
whether they are to be considered successful or not.  We will sometimes refer to
the continuation as the ``future'', since it represents the calculation we want
to perform in the future.

In order to illustrate how backtracking can be implemented using continuation,
let us consider the following invocation of our family database:

\begin{lstlisting}
(query (and (grandparent sam A) (print A)))
\end{lstlisting}

There are two important points to note.  The first is that the general approach
will be a two step process, construct the future that represents the calculation
we want to do, then do it.  The second point is that the details are exceedingly
messy; you should be eternally grateful that it is the computer that is
performing this task, and not you.

To begin, the continuation that represents what it is we want to do after
evaluating the query is the null continuation, an expression that merely returns
true.  In order to try to keep track of the multiple levels of evaluation, let
us write this as follows:

\begin{lstlisting}
(and (grandparent sam A) (print A)) [ true ]
\end{lstlisting}

This says that we want to evaluate the {\sf and} relation, and then do the
calculation given by the bracketed expression.

Consider now the meaning of {\sf and}.  The and expression should evaluate the
first relation, and if successful evaluate the second, and finally if that is
successful evaluate the future given to the original expression.  What then is
the ``future'' of the first relation?  It is simply the second relation and the
original future.  That is, the calculation we want to perform if the first
relation is successful is simply the following:

\begin{lstlisting}
(print A) [ true ]
\end{lstlisting}

We can wrap this in a bracket in order to make a continuation in our form out of
it.  Using {\em this} as the future for the first relation gives us the
following:

\begin{lstlisting}
(grandparent sam A) [ (print A) [ true ] ]
\end{lstlisting}

We are in effect turning the calculation inside out.  We have replaced the
{\sf and} conjunction with a list of expressions to evaluate in the future.

The invocation of the {\sf grandparent} relation causes the expression to be
replaced by the function definition, with the arguments suitably bound to the
parameters.  That is, the effect is the same as:\footnote{We will use textual
    replacement of the parameters by the arguments in our example, although in
    practice the effect is achieved via a level of indirection provided by
    environments, as in all the interpreters we have studied.}

\begin{lstlisting}
(and (parent sam Z) (parent Z A)) [ (print A) [ true ] ]
\end{lstlisting}

We have already analyzed the meaning of the {\sf and} relation.  The future we
want to provide for the first relation is the expression yielded by:

\begin{lstlisting}
(parent Z A) [ (print A) [ true ] ]
\end{lstlisting}

As before, we can expand the invocation of the {\sf parent} relation by
replacing it by its definition, making suitable transformations of the argument
values.

\begin{lstlisting}
(or
    (and (:=: Z alice) (:=: A sally))
    (and (:=: Z sam) (:=: A alice)) )
         [ (print A) [ true ] ]
\end{lstlisting}

The {\sf or} relation should try each alternative in turn, passing it as the
future the continuation passed to the or.  If any is successful we should return
success, otherwise the or should fail.  Thus we can distribute the future to
each clause of the or, and rewrite it as follows:\footnote{The fact that we are
    replacing or by a conditional may seem odd, but the more important point is
    that we have moved the evaluation of the future down to each of the
    arguments to the or expression.}

\begin{lstlisting}
if (and (:=: Z alice) (:=: A sally))
    [ (print A) [ true ] ]
then return true
else if (and (:=: Z sam) (:=: A alice)) )
    [ (print A) [ true ] ]
then return true
else return false
\end{lstlisting}

If we perform the already-defined transformations on the {\sf and} relations we
obtain the following:

\begin{lstlisting}
if (:=: Z alice) [ (:=: A sally) [ (print A) [ true ] ] ]
then return true
else if (:=: Z sam) [ (:=: A alice) [ (print A) [ true ] ] ]
then return true
else return false
\end{lstlisting}

Recall that this was all performed just to construct the continuation for the
first clause in an earlier expression.  Thus the expression we are now working
on is as follows:

\begin{lstlisting}
(parent sam Z) [
    if (:=: Z alice) [ (:=: A sally) [ (print A) [ true ] ] ]
    then return true
    else if (:=: Z sam) [ (:=: A alice) [ (print A) [ true ] ] ]
    then return true
    else return false ]
\end{lstlisting}

We before, we can expand the call on {\sf parent} by its definition:\footnote{I
    warned you about the messy details!}

\begin{lstlisting}
(or
    (and (:=: sam alice) (:=: Z sally))
    (and (:=: sam sam) (:=: Z alice)))
[ if (:=: Z alice) [ (:=: A sally) [ (print A) [ true ] ] ]
    then return true
    else if (:=: Z sam) [ (:=: A alice) [ (print A) [ true ] ] ]
    then return true
    else return false ]
\end{lstlisting}

Once again distributing the future along each argument of the or expression
yields:

\begin{lstlisting}
if (and (:=: sam alice) (:=: Z sally))
    [ if (:=: Z alice) [ (:=: A sally) [ (print A) [ true ] ] ]
    then return true
    else if (:=: Z sam) [ (:=: A alice) [ (print A) [ true ] ] ]
    then return true
    else return false ]
then return true
else if (and (:=: sam sam) (:=: Z alice))
    [ if (:=: Z alice) [ (:=: A sally) [ (print A) [ true ] ] ]
    then return true
    else if (:=: Z sam) [ (:=: A alice) [ (print A) [ true ] ] ]
    then return true
    else return false ]
else return false
\end{lstlisting}

Performing yet one more time the transformations on the {\sf and} relations
yields:

\begin{lstlisting}
if (:=: sam alice) [ (:=: Z sally)
    [ if (:=: Z alice) [ (:=: A sally) [ (print A) [ true ] ] ]
    then return true
    else if (:=: Z sam) [ (:=: A alice) [ (print A) [ true ] ] ]
    then return true
    else return false ] ]
then return true
else if (:=: sam sam) [ (:=: Z alice)
    [ if (:=: Z alice) [ (:=: A sally) [ (print A) [ true ] ] ]
    then return true
    else if (:=: Z sam) [ (:=: A alice) [ (print A) [ true ] ] ]
    then return true
    else return false ]
then return true
else return false
\end{lstlisting}

This is the final continuation that is constructed by the query expression.  The
most important feature of this expression is that it can be evaluated in a
forward fashion, without backtracking.  Having generated it, the next step is
execution.  Contrast this with the description we provided earlier.  First an
attempt is made to unify the symbols {\sf sam} and {\sf alice}.  This fails, and
thus the continuation for the first conditional is ignored.  Next an attempt is
made to unify the symbols {\sf sam} and {\sf sam}.  This is successful, and thus
we evaluate the continuation to the next expression.  The continuation unifies
{\sf Z} and {\sf alice}, binding the left-hand variable to the right-hand
symbol.  The continuation for that expression then trys to unify {\sf Z} and
{\sf alice}, which is successful.  Thus variable {\sf A} is bound to {\sf
    sally}, and is printed.\footnote{Observant readers will have noted that some
    of the conditionals could have been evaluated during the construction of the
    continuation.  This is true, and is an important optimization in real
    systems.}

Having described the general approach our interpreter will follow, we will now
go on to provide the specific details.

Our continuations are built around a new datatype, which we will call the {\sf
    Continuation}.  A continuation should be thought of as an unevaluated
boolean expression.  The continuation performs some action, which may or may not
succeed.  The success of the action is indicated by the boolean value returned.
The class Continuation is shown in Figure~\ref{PrologContiuation}.  The routine
used to invoke a relation is the virtual method {\sf withContinuation}, which
takes as argument the future for the continuation.
%
\includecode{prolog.C}{PrologContiuation}
{The class {\sf Continuation}}

Initially there is nothing we want to do in the future.  So the initial relation
simply ignores its future, does nothing and always succeeds.  In fact, in our
implementation we maintain a global variable called {\sf nothing} to hold this
relation.  You can think of this variable as maintaining the relation $[$ true
$]$.

The simplest relation is the one correspond to the command to print.  When a
print relation is created, the value it will eventually print is saved as part
of the relation.  If the argument passed to the print relation is, following any
indirection, a symbolic value than it is printed out, and the future passed to
the relation is invoked.  If the argument was not a symbol, or if the future
calculation was unsuccessful, then the relation indicates its failure by
returning a zero value.  The code to accomplish this is shown in
Figure~\ref{PrologPrintContinuation}.
%
\includecode{prolog.C}{PrologPrintContinuation}
{The print relation}

Next let us consider the unification relation.  As with printing, the two
expressions representing the elements to be unified are saved when the
unification operator is encountered during the construction of the future.  When
we invoke this relation the two arguments are unified, using the algorithm we
have previously described.  If this unification is successful the relation
attempts to evaluate the future continuation.  Only if both of these are
successful does the relation return one.  If either the unification fails or the
future fails then the binding created by the {\sf unify} procedure is undone and
failure is reported.  (Figure~\ref{PrologUnifyContinuation}).
%
\includecode{prolog.C}{PrologUnifyContinuation}
{The unification relation}

Next consider the {\sf or} relation (Figure~\ref{PrologOrContinuation}).  This
relation takes some number of argument relations.  It tries each in turn,
followed by the future it has been provided with.  If any succeeds then it
returns a true value, otherwise if all fail it returns a failure indication.
%
\includecode{prolog.C}{PrologOrContinuation}
{The {\sf or} relation}

It is in the {\sf or} relation that backtracking occurs, although it is
difficult to tell from the code shown here.  Recall that the unification
algorithm undoes the effect of any assignment if the continuation passed to it
cannot be performed.  Thus the future that is passed to the {\sf or} relation
may be invoked several times before we finally find a sequence of assignments
that works.

The {\sf and} relation is perhaps the most interesting.  To understand this let
us first take the case of only two relations, which we will call {\em rel1} and
{\em rel2}.  Let {\sf f} represent the continuation we wish to evaluate if the
{\sf and} relation is successful.  What then is the future we should pass to the
first relation?  If the first relation is successful, we want to evaluate the
second relation and then the continuation.  Thus the future for the first
relation is the composition of the future for the second relation and the
original continuation.  This can be written as {\sf rel2(f)}, but we must make
it into a continuation, we we create a new datatype called a {\sf
    CompositionContinuation}.  Interestingly, this composition relation ignores
{\em its} continuation, and is merely executed for its side effect.  This is the
future we want to pass to the first relation.  We can generalize this to any
number of arguments.  For example the {\em and} of three arguments should return
the value produced by {\sf rel1([rel2([rel3([f])])])}, and so on.

The composition step is performed by the datatype {\sf ComposeContinuation},
shown in Figure~\ref{PrologComposeContinuation}.  As in our description, when a
composition relation is evaluated it ignores the future it is provided with and
merely returns the first relation provided with the second relation as its
future.  Having defined this, the {\sf and} relation
(Figure~\ref{PrologAndContinuation}) is a simple recursive invocation.
%
\includecode{prolog.C}{PrologComposeContinuation}
{The {\sf compose} continuation}
%
\includecode{prolog.C}{PrologAndContinuation}
{The {\sf and} relation}

You may have noticed that the class {\sf Continuation} is not a subclass of
class {\sf Function}, and yet we have been discussing continuations as if they
were functions.  This is easily explained.  Recall that evaluating a relation in
our approach is a two-step process.  First the relation is constructed, and in
the second step the future is brought to life.  The functional parts of each of
the four relation-building operations are concerned only with the first part of
this task.  These are all trivial functions, shown in
Figure~\ref{PrologUnifyOperation}.
%
\includecode{prolog.C}{PrologUnifyOperation}
{Building the Relations}

The {\sf query} statement is responsible for the construction and execution of
the continuation corresponding to its argument.  The function implementing the
{\sf query} statement is shown in Figure~\ref{PrologQueryStatement}.  A new
environment is created prior to evaluating the arguments so that bindings
created for new variables do not get entered into the global environment.  Then
the continuation is constructed, simply by evaluating the argument.  If this
process is successful, the continuation is then executed, and if the
continuation is successful the symbol {\sf ok} is yielded as the result (and
thus printed by the read-eval-print loop).  If the continuation is not
successful the symbol {\sf not-ok} is generated.
%
\includecode{prolog.C}{PrologQueryStatement}
{Implementation of the query statement}

The initialization function for the prolog interpreter
(Figure~\ref{PrologInitialize}) is one of the shortest we have seen.  It is only
necessary to create the two commands {\sf define} and {\sf query}, and the four
relational-building operations.
%
\includecode{prolog.C}{PrologInitialize}
{Initialization of the Prolog interpreter}

\chapter*{Possible Future Changes}

The following list represents a few of the ideas that occurred to me as I was
developing these interpreters for how things might be done differently.  These
are presented in no particular order.  (Nor as any particularly grave criticism
of the Kamin interpreters - I still think the book as a whole is very good).
\begin{itemize}
    \item The C++ versions of the interpreters have an annoying habit of dumping
    core when an error occurs.  Need to track this down and fix it.
    \item I would remove the while statement from the chapter 2 lisp
    interpreter.  Students who do not have previous experience with Lisp often
    have a difficult time learning to program in a recursive fashion.  For them
    the while statement is a crutch, and without it they would be forced to use
    the more Lisp-like features of the language.
    \item I would add functionals (called operators in APL) to chapter 3.
    Specifically I would make reduction take the function as an argument, and
    add inner and outer product.  This would allow an easier transition to
    functional programming in the next section.
    \item I might be tempted to add a chapter before chapter 3 on Setl.  This is
    another example of a language using large values, and allows a new and
    different problem domain to be discussed (namely logic).
    \item It would be nice to add call/cc to the scheme interpreter, but I don't
    exactly see how to do this right now.  This is not quite as critical now
    that the Prolog interpreter uses continuations for its execution.
    \item I would remove the keyword ``rep'' from the CLU syntax, as it is
    unnecessary and its elimination simplifies the implementation.
\end{itemize}


\end{document}
