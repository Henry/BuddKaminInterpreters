\chapter{The Lisp Interpreter}

The interpreter for Lisp differs only slightly from that of Chapter one.  The
reader/parser is modified so as to recognize quoted constants, two new global
variables ({\sf T} and {\sf nil}) are added, and a number of new value-ops are
defined.  In all other respects it is the same.  Figure~\ref{chap2hier} shows
the class hierarchy for the expression classes added in chapter 2.

\setlength{\unitlength}{5mm}
\begin{figure}
\begin{picture}(16,5)(-4,-3)
\put(-3.5,0){\sf Expression}
\put(0,0.2){\line(1,0){1}}
\put(1,0){\sf Function}
\put(0,0.2){\line(1,-1){1}}
\put(1,-1){\sf QuotedConstant}
\put(4,0.2){\line(1,0){1}}
\put(5,0){\sf BinaryFunction}
\put(4,0.2){\line(1,3){1}}
\put(5,3){\sf UnaryFunction}
\put(9.5,3.2){\line(1,0){1}}
\put(10.5,3){\sf BooleanUnary}
\put(9.5,0.2){\line(1,1){1}}
\put(10.5,1){\sf IntegerBinaryFunction}
\put(9.5,0.2){\line(1,-1){1}}
\put(10.5,-1){\sf BooleanBinaryFunction}
\end{picture}
\caption{Classes added in Chapter Two}\label{chap2hier}
\end{figure}

\section{The Lisp reader}

The Lisp reader is created by subclassing from the base class {\sf Reader}
(Figure~\ref{LispReader}).  The only change is to modify the method {\sf
    readExpression} to check for leading quote marks.  If no mark is found,
execution is as in the default case.  If a quote mark is found, the character
pointer is advanced and the following expression is turned into a quoted
constant.  Note that no checking is performed on this expression.  This permits
symbols, even separators, to be treated as data.  That is, '; is a quoted
symbol, even though the semicolon itself is not a legal symbol.
%
\includecode{lisp.h}{LispReader}
{The Lisp reader/parser}
%
\includecode{lispPrimitives.C}{LispReaderImpl}
{The Lisp reader/parser implementation}

To create quoted constants it is necessary to introduce a new type of
expression.  When an instance of class {\sf QuotedConst} is evaluated, it simply
returns its (unevaluated) data value.

\section{Value-ops}

In addition to adding a number of new value-ops, the Lisp interpreter modifies
the meaning of a few of the Chapter 1 functions.  For example the relational
operators must now return the values {\sf T} or {\sf nil}, and not 1 and 0
values.  Similarly the meaning of {\em true} and {\em false} used by the {\sf
    if} and {\sf while} statements is changed.  Finally the equality testing
function ($=$) must now recognize both symbols and integers.

\subsection{Relationals}

Figure~\ref{EqualFunction} shows the revised definition of the equality testing
function, which now must be prepared to handle symbols and well as integers.
%
\includecode{lispPrimitives.C}{EqualFunction}
{The revised Definition of the equality function}

Implementation of the boolean binary functions is simplified by the introduction
of a class {\sf BooleanBinaryFunction} (Figure~\ref{BooleanBinaryFunction}).
This class decodes the two integer arguments and invokes a further method to
determine the boolean result.  Based on this result either the value of the
global symbol representing true or the symbol representing false is returned.
%
\includecode{function.h}{BooleanBinaryFunction}
{Class returning boolean results from relationals}
%
\includecode{function.C}{BooleanBinaryFunctionApply}
{{\sf BooleanBinaryFunction} apply function}

Finally Figure~\ref{LispIsTrue} shows the revised function used by {\sf if} and
{\sf while} statements to determine the truth or falsity of their condition.
Unlike in Chapter 1, where 0 and 1 were used to represent true and false, here
{\sf nil} is used as the only false value.
%
\includecode{lisp.C}{LispIsTrue}
{Specialised {\sf isTrue} for lisp}

\subsection{Car, Cdr and Cons}

{\sf car} and {\sf cdr} are implemented as simple unary functions
(Figure~\ref{CarCdrCons}), and {\sf cons} is a simple binary function that
creates a new {\sf ListNode} out of its two arguments.
%
\includecode{lispPrimitives.C}{CarCdrCons}
{{Implementation of {\sf car}, {\sf cdr} and {\sf cons}}}

\footnote{A matter for debate is whether Cons should give an error if the second
    argument is not a list.  Real Lisp doesn't care; but also uses a different
    format for printing such lists.  Our interpreter prints such as lists
    exactly as if the second argument had been a list containing the element.}

\subsection{Predicates}

The implementation of the predicates {\sf number?}, {\sf symbol?}, {\sf list?}
and {\sf null?} is simplified by the creation of a class {\sf BooleanUnary}
(Figure~\ref{BooleanUnary}), subclassing {\sf UnaryFunction}.  As with the
integer functions implemented in chapter 1, instances of {\sf BooleanUnary}
maintain as part of their state a function that takes an expression and returns
an integer (that is, boolean) value.  Thus for each predicate it is only
necessary to write a function which takes the single argument and returns a
true/false indication.
%
\includecode{lisp.h}{BooleanUnary}
{The class {\sf BooleanUnary}}
%
\includecode{lispPrimitives.C}{BooleanUnaryApply}
{{\sf BooleanUnary} apply function}

\section{Initialization of the Lisp Interpreter}

Figure~\ref{LispInitialize} shows the initialization method for the Lisp
interpreter.
%
\includecode{lisp.C}{LispInitialize}
{Initialization of the Lisp interpreter}
