\chapter{The SASL interpreter}\label{sasl}

The SASL interpreter is largely constructed by removing features from the Scheme
interpreter, such as while loops and so on, and changing the implementation of
the {\sf cons} function to add delayed evaluation.  Figure~\ref{saslhier} shows
the class hierarchy for the classes added in this chapter.

\setlength{\unitlength}{5mm}
\begin{figure}
    \begin{picture}(25,10)(-4,-5)
        \put(-3.5,0){\sf Expression}
        \put(0,0.2){\line(1,0){1}}
        \put(1,0){\sf Function}
        \put(0,0.2){\line(1,-2){1}}
        \put(1,-2){\sf Thunk}
        \put(4,0.2){\line(1,-2){1}}
        \put(5,-2){\sf UserFunction}
        \put(9.5,-1.8){\line(1,0){1}}
        \put(10.5,-2){\sf LazyFunction}
        \put(4,0.2){\line(1,2){1}}
        \put(5,2){\sf SaslConsFunction}
        \put(4,0.2){\line(1,0){1}}
        \put(5,0){\sf LambdaFunction}
    \end{picture}
    \caption{Class Hierarchy for expressions in the SASL interpreter}
    \label{saslhier}
\end{figure}

\subsection{Thunks}

Delayed evaluation is provided by adding a new expression type, called the {\em
    thunk}.  Figure~\ref{SASLThunk} shows the data structure used to represent
this type of value.  Every thunk maintains a boolean value indicating whether
the thunk has been evaluated yet, an expression (representing either the
unevaluated or evaluated expression, depending upon the state of the boolean
flag), and a context in which the expression is to be evaluated.  Thunks print
either as three dots, if they have not yet been evaluated, or as the printed
representation of their value, if they have.
%
\includecode{sasl.C}{SASLThunk}
{Definition of Thunks}
%
\includecode{sasl.C}{SASLThunkTouch}
{Definition of Thunk touch}

Here we finally see an overridden definition for the method {\em touch}.  You
will recall that this method was defined in Chapter 1, and that all other
expressions merely return their value as the result of this expression.  Thunks,
on the other hand, will evaluate themselves if touched, and then return their
new evaluated result.  With the addition of this feature many of the definitions
we have presented in earlier chapters, such as the definitions of {\sf car} and
{\sf cdr}, hold equally well when given thunks as arguments.

Since thunks can represent lists, symbols, integers and so on, the predicate
methods {\sf isSymbol} and the like must be redefined as well.  If the thunk
represents an evaluated value, these simply return the result of testing that
value (Figure~\ref{SASLThunkPredicates}).
%
\includecode{sasl.C}{SASLThunkPredicates}
{Thunk predicates}
%
\includecode{sasl.C}{SASLThunkEval}
{Thunk eval}

\section{Lazy Cons}

The SASL cons function differs from the Scheme version in producing a list node
containing a pair of thunks, rather than a pair of values
(Figure~\ref{SaslConsFunction}).  Class {\sf SaslConsFunction} must now be a
subclass of {\sf Function} and not {\sf BinaryFunction}, because it must grab
its arguments before they are evaluated.  Thus it must itself check to see that
only two arguments are passed to the function.
%
\includecode{sasl.C}{SaslConsFunction}
{The Sasl Lazy Cons function}

\section{Lazy User Functions}

User defined functions must be provided with lazy evaluation semantics as well.
This is accomplished by defining a new class {\sf LazyFunction}
(Figure~\ref{SASLLazyFunction}).  Lazy functions act just like user functions
from previous chapters, only they do not evaluate their arguments.  Thus the
function body is evaluated by the method {\sf apply}, rather than passing the
evaluated arguments on to the method {\sf applyWithArgs}.  The lambda function
from the previous chapter is modified to produce an instance of {\sf
    LazyFunction}, rather than {\sf UserFunction}.
%
\includecode{sasl.C}{SASLLazyFunction}
{The implementation of lazy functions}
