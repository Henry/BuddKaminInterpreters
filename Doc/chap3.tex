\chapter{The APL Interpreter}

My version of the APL interpreter differs somewhat from that provided by Kamin:
\begin{itemize}
    \item My version will recognize arbitrary rank (dimension) arrays, not
    simply scalar, vector and two dimensional arrays.  (Although currently it is
    only able to print those three types).
    \item The C++ version of the interpreter recognizes vector constants without
    the necessity for quoting them, as in (resize (3 4) (indx 12)).
    \item I have eliminated the if and while statements, thus forcing
    programmers into a more ``APL'' style of thought.
    \item My version of catenation works now for values of arbitrary
    dimensionality.  (Transpose and print are the only two functions that limit
    the dimensionality of their arguments).
\end{itemize}

Despite the APL interpreter being larger than any other interpreter, I think
that the addition of a few more functions could give the student an even better
feel for the language, as well as providing a smooth transition to functional
programming.  Specifically, I think reduction should be defined as a functional,
and inner and outer products added as operations.  I have not done this as yet,
however.

Figure~\ref{aplhier} shows the class hierarchy for the classes introduced in
this chapter.

\setlength{\unitlength}{5mm}
\begin{figure}
    \begin{picture}(25,10)(-4,-5)
        \put(-3.5,0){\sf Expression}
        \put(0,0.2){\line(1,0){1}}
        \put(1,0){\sf Function}
        \put(0,0.2){\line(1,-2){1}}
        \put(1,-2){\sf APLValue}
        \put(4,0.2){\line(1,-2){1}}
        \put(5,-2){\sf BinaryFunction}
        \put(9.5,-1.8){\line(1,0){1}}
        \put(10.5,-2){\sf APLBinaryFunction}
        \put(4,0.2){\line(1,2){1}}
        \put(5,2){\sf UnaryFunction}
        \put(9.5,2.2){\line(1,0){1}}
        \put(10.5,2){\sf APLUnaryFunction}
        \put(16,2.2){\line(1,0){1}}
        \put(17,2){\sf RavelFunction}
        \put(16,2.2){\line(1,1){1}}
        \put(17,3){\sf ShapeFunction}
        \put(16,2.2){\line(1,2){1}}
        \put(17,4){\sf APLReduction}
        \put(16,2.2){\line(1,-1){1}}
        \put(17,1){\sf IndexFunction}
        \put(16,2.2){\line(1,3){1}}
        \put(17,5){\sf TransposeFunction}
        \put(16,-1.8){\line(1,0){1}}
        \put(17,-2){\sf RestructFunction}
        \put(16,-1.8){\line(1,1){1}}
        \put(17,-1){\sf CompressFunction}
        \put(16,-1.8){\line(1,-1){1}}
        \put(17,-3){\sf APLScalarFunction}
        \put(16,-1.8){\line(1,-2){1}}
        \put(17,-4){\sf CatenationFunction}
        \put(16,-1.8){\line(1,-3){1}}
        \put(17,-5){\sf SubscriptionFunction}
    \end{picture}
    \caption{The APL interpreter class hierarchy}\label{aplhier}
\end{figure}

\section{APL Values}

The APL interpreter manipulates APL values, which are defined by the data type
{\sf APLValue} (Figure~\ref{APLValue}).  An APL value represents a integer
rectilinear array.  Internally, such a value is represented by a list that
maintains the shape (extent along each dimension) and a vector of integer
values.  The length of the shape list provides the rank (dimensionality) of the
data value.  The product of the values in the shape indicates the number of
elements in the array, except in the case of scalar values, which have an empty
shape array.
%
\includecode{apl.C}{APLValue}
{The Representation for APL Values}

APL values are stored in what is called {\em ravel-order}.  This is what in some
other languages is called row-major order.

The methods defined for APL values can be used to determine the number of
elements contained in the structure ({\sf size}), obtain the shape of the value
({\sf shape}), obtain the shape at any given dimension ({\sf shapeAt}), obtain
the value at any given ravel-order position ({\sf at}), and finally change the
value at any position ({\sf atPut}).

\section{The APL Reader}

The APL reader is modified so that individual scalar values and vectors of
integers are recognized as APL values.  The class definition for {\sf APLreader}
is shown in Figure~\ref{APLreader}, and the code for the two auxiliary functions
in the next figure.
%
\includecode{apl.C}{APLreader}
{The APL reader}
%
\includecode{apl.C}{readAPLscalar}
{The APL reader functions}

\section{APL Functions}

The implementation of the APL functions is simplified by the addition of two
auxiliary classes, {\sf APLUnary} and {\sf APLBinary}.  In addition to checking
that the right number of arguments are provided to a function application, these
check to insure that the arguments are APL values \footnote{A largely gratuitous
    move, since the user has no way of creating anything other than an APL
    value.  Still, it doesn't do any harm to be careful.} and invoke yet another
virtual function {\sf applyOp}, to perform the actual calculation.

\subsection{Scalar Functions}

By far the largest class of APL functions are the so-called {\em scalar
    functions}.  These are the conventional arithmetic and logical functions,
such as addition and multiplication, extended in the natural way to arrays.  The
only complication in the implementation of these values concerns what is called
{\em scalar extension}.  That is, a scalar value can be used as either the left
or right argument to a scalar function, and it is treated as if it were an
entire array of the correct dimensionality to match the other argument.  Since
scalar extension can occur with either the left or right argument, the code for
scalar functions divides naturally into three cases.

Scalar functions are implemented using a single class by making use, as we have
done before, of an instance variable that contains a pointer to a integer
function that generates an integer result.  The class {\sf APLscalarFunction}
and the method {\sf applyOp} are shown in Figure~\ref{APLScalarFunctionApply}.
Note that the same functions used in the previous interpreters can be used in
the construction of the APL scalar functions.
%
\includecode{apl.C}{APLScalarFunctionApply}
{APL Scalar Functions}

\subsection{Reduction}

For each scalar function there is an associated reduction function.\footnote{The
    statement is true of real APL.  The Kamin interpreters do not implement
    reductions with relational operators, which are, however, not particularly
    useful.}  Reduction in these interpreters always occurs along the last
dimension.  Thus to compute the size of a new value is suffices to remove the
last dimension value.  This also simplifies the generation of the new values,
since the argument array can be processed in units as long as the final
dimension.  As with the scalar functions, there is one class defined for all the
reductions, with each instance of this class maintaining the particular scalar
function being used for the reduction operations.  Figure~\ref{APLReduction}
shows the code used in computing the APL reduction function.
%
\includecode{apl.C}{APLReduction}
{Implementation of the APL reduction function}

\subsection{Compression}

Compression, like reduction, operates on the last dimension of a higher order
array, changing its extent to that of the number of one elements in the
left-argument vector.  The length of the left argument vector must match the
extent of the last dimension of the right argument.  The compression function
(Figure~\ref{APLCompressionFunctionApply}) first computes the number of one
elements in the left argument, then iterates over the right argument generating
the new values.
%
\includecode{apl.C}{APLCompressionFunctionApply}
{The Compression function}

\subsection{Shape and Reshape}

The {\sf shape} function merely copies the size on its argument into a new APL
value.  The reshape function ({\sf restruct}) generates a new value with a size
given by the left argument, which must be a vector, using elements from the
right argument, recycling over the ravel ordering of the right argument multiple
times if necessary.  The implementation of these functions is shown in
Figure~\ref{APLShapeFunctionApply} and ref{APLRestructFunctionApply}.
%
\includecode{apl.C}{APLShapeFunctionApply}
{The shape function}
%
\includecode{apl.C}{APLRestructFunctionApply}
{The reshape function}

\subsection{Ravel and Index}

The ravel function (Figure~\ref{APLRavelFunctionApply}) merely takes an argument
of arbitrary dimensionality and returns the values as a vector.  The index
function (called iota in real APL) (Figure~\ref{APLIndexFunctionApply}) takes a
scalar value and returns a vector of numbers from 1 to the argument value.
%
\includecode{apl.C}{APLRavelFunctionApply}
{Ravel}
%
\includecode{apl.C}{APLIndexFunctionApply}
{Index}

\subsection{Catenation}

The catenation function joins two arrays along their last dimension.  They must
match in all other dimensions.  To build the new result first a row from the
first array is copies into the final array, then a row from the second array,
then another row from the first, followed by another row from the second, and so
on until all rows from each argument have been used.
%
\includecode{apl.C}{APLCatenationFunctionApply}
{Implementation of the Catenation function}

\subsection{Transpose}

While real APL defines transpose for arbitrary dimension arrays, the transpose
presented here works only for arrays of dimension two or less.  For vector and
scalars the transpose does nothing.  Thus the only code required
(Figure~\ref{APLTransposeFunctionApply}) is to take the transpose of a two
dimensional array.
%
\includecode{apl.C}{APLTransposeFunctionApply}
{The Transpose Function}

\subsection{Subscription}

The Pascal interpreter provided by Kamin applies subscription to the first
dimension of a multidimension value.  In order to be consistent with the other
functions, my version does subscription along the last dimension.  Neither is
exactly the same as the real APL version.  The subscription code is shown in
Figure~\ref{APLSubscriptFunction}.
%
\includecode{apl.C}{APLSubscriptFunction}
{The Subscription function}

\section{Initialization of the APL interpreter}

The initialization code for the APL interpreter is shown in
Figure~\ref{APLInitialize}.
%
\includecode{apl.C}{APLInitialize}
{APL interpreter initialization}
