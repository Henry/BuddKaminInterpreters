\chapter{The CLU interpreter}

The CLU interpreter is created by introducing a new datatype, the cluster,
and three new types of functions.  Constructors create new instances of a
cluster, selectors access a portion of a cluster state, and modifiers change
a portion of a cluster state.  Figure~\ref{cluhier} shows the class
hierarchy for the classes added in this chapter.

\setlength{\unitlength}{5mm}
\begin{figure}
    \begin{picture}(25,10)(-4,-5)
        \put(-3.5,0){\sf Expression}
        \put(0,0.2){\line(1,0){1}}
        \put(1,0){\sf Function}
        \put(0,0.2){\line(1,-2){1}}
        \put(1,-2){\sf Cluster}
        \put(4,0.2){\line(1,-2){1}}
        \put(5,-2){\sf BinaryFunction}
        \put(9.5,-1.8){\line(1,0){1}}
        \put(10.5,-2){\sf Modifier}
        \put(4,0.2){\line(1,2){1}}
        \put(5,2){\sf UnaryFunction}
        \put(9.5,2.2){\line(1,0){1}}
        \put(10.5,2){\sf Selector}
        \put(4,0.2){\line(1,0){1}}
        \put(5,0){\sf Constructor}
        \put(4,0.2){\line(1,4){1}}
        \put(5,4){\sf ClusterDef}
    \end{picture}
    \caption{Class Hierarchy for the CLU interpreter}
    \label{cluhier}
\end{figure}

\section{Clusters}

A cluster simply encapsulates a series of names and values, hiding them from
normal examination.  The most natural way to do this is for a cluster to
maintain an environment (Figure~\ref{CLUCluster}).  The predicate {\sf
    isCluster} returns this environment value.
%
\includecode{clu.C}{CLUCluster}
{The definition of a cluster value}

To create a cluster requires a constructor function.  The constructor is
provided with a list of names of the elements in the internal representation of
the cluster, and simply insures that the arguments it is provided with match in
number of the names it maintains.

\section{Selectors and Modifiers}

To access or modify the elements of a constructor requires functions called
selectors or modifiers.  Each of these maintain as their state the name of the
field they are responsible for.  When invoked with a constructor, the access or
change their given field.
%
\includecode{clu.C}{CLUSelectorModifier}
{Selectors and Modifiers for clusters}

\section{Defining clusters}

It thus remains only to give the (rather lengthy) definition of the function
that generates constructor information from the textual description.  (We do not
say generates clusters themselves, for that is the responsibility of the
constructor functions).  This function is shown in Figure~\ref{CLUClusterDef}.
It rips apart a cluster definition and does the right things (need a better
description here, but I don't have time to write it now).  (Need to point out
that cluster functions have an internal and an external name, and these are put
of different environments).  (I suppose an alternative would have been to
introduce a new datatype for two part names, which when evaluated would look up
their second part in the cluster provided by their first part).
%
\includecode{clu.C}{CLUClusterDef}
{The cluster recognition function}
