\chapter{The Scheme Interpreter}

After all the code required to generate the APL interpreter of Chapter 3, the
Scheme interpreter is simplicity in itself.  Of course, this has more to do with
the similarity of Scheme to the basic Lisp interpreter of Chapter 2 than with
any differences between APL and Scheme.

To implement Scheme it is only necessary to provide an implementation of the
lambda function.  This is accomplished by the class {\sf Lambda}, shown in
Figure~\ref{SchemeLambdaFunction}.  The actual implementation of lambda uses the
same class {\sf UserFunction} we have seen in previous chapters.
%
\includecode{scheme.C}{SchemeLambdaFunction}
{The class Lambda}

Initialization of the Scheme interpreter differs slightly from the code used to
initialize the Lisp interpreter (Figure~\ref{SchemeInitialize}).  The {\sf
    define} command is no longer recognized, having been replaced by the {\sf
    set}/{\sf lambda} pair.  The built-in arithmetic functions are now considred
to be global symbols, and not value-ops.  Indeed, there are no comands or
value-ops in this language.
%
\includecode{scheme.C}{SchemeInitialize}
{Initialization of the Scheme Interpreter}
